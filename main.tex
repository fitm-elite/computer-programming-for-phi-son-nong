\documentclass[12pt,a4paper]{article}

% Page setup
\usepackage[a4paper,top=1in,bottom=1in,left=0.5in,right=0.5in]{geometry}
\usepackage{setspace}
\onehalfspacing

\usepackage{etoolbox}
\usepackage[nonumberlist]{glossaries}
\usepackage{hyperref}
\usepackage{graphicx}
\usepackage{float}
\usepackage{xcolor}
\usepackage{colortbl}

% Table packages
\usepackage{array}
\usepackage{booktabs}
\usepackage{longtable}
\usepackage{tabularx}
\usepackage{multirow}
\usepackage{multicol}

% List formatting
\usepackage{enumitem}

% Dotted lines package
\usepackage{dashrule}

% Code listings package for syntax highlighting
\usepackage{listings}
\usepackage{xcolor}
\usepackage{tcolorbox}
\tcbuselibrary{listings,skins}

% Thai language support
\usepackage{xunicode}
\usepackage{xltxtra}

% Optimized fonts for Thai language - using Medium weight
\setmainfont[
    Path=./font/Sarabun/,
    UprightFont=*-Regular,
    BoldFont=*-Medium,
    ItalicFont=*-Italic,
    BoldItalicFont=*-MediumItalic,
    Scale=0.90,
    Ligatures=TeX,
    WordSpace=1.2,
    PunctuationSpace=1.1
]{Sarabun}

\newfontfamily\thaifont[
    Path=./font/Sarabun/,
    UprightFont=*-Regular,
    BoldFont=*-Medium,
    ItalicFont=*-Italic,
    BoldItalicFont=*-MediumItalic,
    Scale=0.90,
    Ligatures=TeX,
    WordSpace=1.2,
    PunctuationSpace=1.1,
    Script=Thai,
    Language=Thai
]{Sarabun}

% Light weight font family optimized for Thai
\newfontfamily\thailightfont[
    Path=./font/Sarabun/,
    UprightFont=*-Light,
    BoldFont=*-Medium,
    ItalicFont=*-LightItalic,
    BoldItalicFont=*-MediumItalic,
    Scale=0.90,
    Ligatures=TeX,
    WordSpace=1.2,
    PunctuationSpace=1.1,
    Script=Thai,
    Language=Thai
]{Sarabun}

% Font commands for easy switching
\newcommand{\textlight}[1]{{\thailightfont #1}}
\newcommand{\thailight}{\thailightfont}

% Thai typography optimization commands
\newcommand{\optimalthai}[1]{{\thaifont #1}}

% Monospace font for Thai in code blocks
\newfontfamily\thaifonttt[
    Path=./font/Sarabun/,
    UprightFont=*-Regular,
    BoldFont=*-Medium,
    Scale=0.9,
    Ligatures=TeX,
    Script=Thai,
    Language=Thai
]{Sarabun}

% Additional packages for better Thai support
\usepackage{polyglossia}
\setdefaultlanguage{thai}
\setotherlanguage{english}

% Thai line breaking settings - optimized
\XeTeXlinebreaklocale "th"
\XeTeXlinebreakskip = 0pt plus 2pt minus 1pt

% Thai typography enhancements
\frenchspacing 
\tolerance=1000
\emergencystretch=3em

% Thai punctuation spacing
\XeTeXinterchartokenstate = 1

% Better Thai paragraph spacing
\setlength{\parskip}{0.5\baselineskip plus 0.2\baselineskip minus 0.1\baselineskip}
\setlength{\parindent}{2em}

% Custom section numbering with period and reduced spacing
\renewcommand{\thesection}{\arabic{section}.}
\renewcommand{\thesubsection}{\arabic{section}.\arabic{subsection}.}
\renewcommand{\thesubsubsection}{\arabic{section}.\arabic{subsection}.\arabic{subsubsection}}

% Custom dotted rule command
\newcommand{\dotrule}[1]{\hdashrule{#1}{0.6pt}{1pt}}

% Flowchart and diagram packages
\usepackage{tikz}
\usepackage{tikz-qtree}
\usetikzlibrary{shapes.geometric, arrows, positioning, fit, calc}

% Define flowchart styles
\tikzstyle{startstop} = [rectangle, rounded corners, minimum width=3cm, minimum height=1cm, text centered, draw=black, fill=red!30]
\tikzstyle{process} = [rectangle, minimum width=3cm, minimum height=1cm, text centered, draw=black, fill=blue!30]
\tikzstyle{decision} = [diamond, minimum width=1.5cm, minimum height=1cm, text centered, draw=black, fill=green!30]
\tikzstyle{io} = [trapezium, trapezium left angle=70, trapezium right angle=110, minimum width=2cm, minimum height=1cm, text centered, draw=black, fill=yellow!30]
\tikzstyle{arrow} = [thick,->,>=stealth]

% Code styling configuration
\definecolor{codegreen}{rgb}{0,0.6,0}
\definecolor{codegray}{rgb}{0.5,0.5,0.5}
\definecolor{codepurple}{rgb}{0.58,0,0.82}
\definecolor{backcolour}{rgb}{1,1,1}

% Monospace font for code
\newfontfamily\codefont{Menlo}[
    Scale=0.85
]

% Python code style
\lstdefinestyle{python}{
    backgroundcolor=\color{backcolour},   
    commentstyle=\color{codegreen},
    keywordstyle=\color{magenta},
    numberstyle=\tiny\color{codegray},
    stringstyle=\color{codepurple},
    basicstyle=\codefont\footnotesize,
    breakatwhitespace=false,         
    breaklines=true,                 
    captionpos=b,                    
    keepspaces=true,                 
    numbers=left,                    
    numbersep=10pt,            
    xleftmargin=10pt,
    showspaces=false,                
    showstringspaces=false,
    showtabs=false,                  
    tabsize=2,
    columns=fixed,
    basewidth=0.6em,
    language=Python
}

% Custom code box using tcolorbox
\newtcolorbox{codebox}[2][]{
    colback=white!5!white,
    colframe=gray!75!black,
    fonttitle=\bfseries,
    title=#2,
    top=0pt,
    bottom=0pt,
    boxsep=0pt,
    toptitle=0pt,
    bottomtitle=0pt,
    #1
}

% Custom result box for output display
\newtcolorbox{resultbox}[2][]{
    colback=white!5!white,
    colframe=green!75!black,
    fonttitle=\bfseries,
    title=#2,
    top=3pt,
    bottom=3pt,
    left=4pt,
    #1
}

% Custom find answer box for output display
\newtcolorbox{findanswerbox}[2][]{
    colback=white!5!white,
    colframe=gray!75!black,
    fonttitle=\bfseries,
    title=#2,
    #1
}

% Custom exercise box for exercises
\newtcolorbox{exercisebox}[2][]{
    colback=white!5!white,
    colframe=gray!75!black,
    fonttitle=\bfseries,
    title=#2,
    #1
}

% Custom notice box for important messages
\newtcolorbox{noticebox}[2][]{
    colback=yellow!5!white,
    colframe=magenta!90!black,
    fonttitle=\bfseries,
    title=#2,
    left=6pt,
    right=6pt,
    top=6pt,
    bottom=6pt,
    boxrule=2pt,
    arc=3pt,
    auto outer arc,
    #1
}

% Custom notice box for important messages
\newtcolorbox{answerbox}[2][]{
    colback=yellow!5!white,
    colframe=magenta!50!black,
    fonttitle=\bfseries,
    title=#2,
    left=6pt,
    right=6pt,
    top=6pt,
    bottom=6pt,
    boxrule=2pt,
    arc=3pt,
    auto outer arc,
    #1
}

\begin{document}

% Title Page
\begin{titlepage}
    \centering
    \vspace*{2cm}
    
    {\Huge\bfseries {คอมพิวเตอร์โปรแกรมมิ่ง (โครงการพี่ติวน้อง)}\par}
    {\textlight{เพื่อส่งเสริมการมีส่วนร่วมของรุ่นพี่ในการถ่ายทอดความรู้แก้รุ่นน้อง เพื่อเสริมความเข้าใจในรายวิชา}\par}
    \vspace{2cm}

    {\textlight{\textcolor{darkgray}{*ยงยุทธ ชวนขุนทด, เมธัส ทองจันทร์, ศุภกร จิรศิริวรกุล, มชัญชยา ประยูรมณีรัตน์}}\par}
    {\normalfont ประจำวันที่ 17 กรกฎาคม พ.ศ. 2568\par}
    
    \vfill
    
    {\thailight ภาควิชาเทคโนโลยีสารสนเทศ\par}
    {\thailight มหาวิทยาลัยเทคโนโลยีพระจอมเกล้าพระนครเหนือ วิทยาเขตปราจีนบุรี\par}
\end{titlepage}

% Table of Contents
\tableofcontents
\newpage

\section{ชนิดโฟลวชาร์ต (Flowchart Types)}

\begin{figure}[H]
\centering
\begin{tikzpicture}[node distance=2cm]

% Start or Stop symbol
\node (start) [startstop] {Start or Stop};
\node (startd1) [right of=start, xshift=1.1cm] {\textbf{Oval}: Start/End};

% Process symbol
\node (process) [process, below of=start] {Process};
\node (processd1) [right of=process, xshift=2cm] {\textbf{Rectangle}: Process/Action};

% Output or Input symbol
\node (io) [io, below of=process] {I/O};
\node (iod1) [right of=io, xshift=2cm] {\textbf{Trapezium}: Input/Output};

% Decision symbol
\node (decision) [decision, below of=io] {Decision?};
\node (desisiond1) [right of=decision, xshift=1.5cm] {\textbf{Diamond}: Decision};

\end{tikzpicture}
\caption{ตัวอย่างโฟลวชาร์ตพื้นฐาน}
\end{figure}

\hspace{1cm}\textlight{โฟลวชาร์ต (Flowchart) เป็นเครื่องมือที่ใช้ในการแสดงลำดับขั้นตอนของกระบวนการหรืออัลกอริธึมในรูปแบบกราฟิก โฟลวชาร์ตประกอบด้วยสัญลักษณ์ต่างๆ ที่แสดงถึงการเริ่มต้นหรือสิ้นสุด (Start/Stop), การดำเนินการ (Process), การตัดสินใจ (Decision), และการป้อนข้อมูลหรือแสดงผล (Input/Output) การใช้โฟลวชาร์ตช่วยให้เข้าใจและวางแผนการเขียนโปรแกรมได้ง่ายขึ้น}

\hspace{1cm}\textlight{ในแต่ละสัญลักษณ์จะถูกเชื่อมด้วย ลูกศร (Arrow) เพื่อแสดงทิศทางของการไหลของข้อมูลหรือการดำเนินการในโฟลวชาร์ต การใช้โฟลวชาร์ตช่วยให้สามารถวางแผนและออกแบบโปรแกรมได้อย่างมีประสิทธิภาพ และยังช่วยในการสื่อสารแนวคิดกับผู้อื่นได้ง่ายขึ้น}

\vspace{6cm}

\section{ตัวแปรและชนิดข้อมูล (Variables and Data Types)}
\hspace{1cm}\textlight{สิ่งสำคัญในการเขียนโปรแกรมคือการเข้าใจตัวแปรและชนิดข้อมูล ซึ่งเป็นพื้นฐานของการจัดเก็บและจัดการข้อมูลในโปรแกรม ตัวแปรใช้เพื่อเก็บข้อมูลที่สามารถ เปลี่ยนแปลง (Mutable) และ ไม่สามารถเปลี่ยนแปลง (Immutable) ได้ในระหว่างการทำงานของโปรแกรม และชนิดข้อมูลกำหนดประเภทของข้อมูลที่ตัวแปรนั้นสามารถเก็บได้}

\subsection{ตัวแปร (Variables)}
\hspace{1cm}\textlight{ตัวแปรเป็นชื่อที่ใช้เพื่ออ้างถึงข้อมูลที่เก็บอยู่ในหน่วยความจำของคอมพิวเตอร์ ตัวแปรสามารถเปลี่ยนแปลงค่าได้ในระหว่างการทำงานของโปรแกรม และยังมีตัวแปรที่ไม่สามารถเปลี่ยนแปลงค่าได้ (Immutable) เช่น ค่าคงที่ (Constants) ตัวแปรในภาษาโปรแกรมต่างๆ อาจมีรูปแบบการประกาศที่แตกต่างกัน}

\begin{codebox}{(ตัวอย่าง) ตัวแปรใน Python}
\begin{lstlisting}[style=python]
x = 10
y = 20

print(x, y)
\end{lstlisting}
\end{codebox}

\begin{resultbox}{ผลลัพธ์}
\begin{verbatim}
10 20
\end{verbatim}
\end{resultbox}

\vspace{0.1cm}

\begin{figure}[H]
\centering
\begin{tikzpicture}[node distance=2cm]

\node (start) [startstop] {Start};
\node (x) [process, below of=start] {x = 10};
\node (y) [process, below of=x] {y = 20};
\node (output) [io, below of=y] {x, y};
\node (stop) [startstop, below of=output] {Stop};

\draw [arrow] (start) -- (x);
\draw [arrow] (x) -- (y);
\draw [arrow] (y) -- (output);
\draw [arrow] (output) -- (stop);

\end{tikzpicture}
\caption{ตัวอย่าง Flowchart ของการประกาศตัวแปร}
\end{figure}

\subsection{ชนิดข้อมูล (Data Types)}

\textlight{ชนิดข้อมูล (Data Types) เป็นการกำหนดประเภทของข้อมูลที่ตัวแปรนั้นสามารถเก็บได้ ชนิดข้อมูลที่พบบ่อย ได้แก่}

\begin{itemize}
    \item \textbf{Integer (int)}: \textcolor{darkgray}{ตัวเลขจำนวนเต็ม เช่น 1, 2, -3}
    \item \textbf{Float (float)}: \textcolor{darkgray}{ตัวเลขทศนิยม เช่น 3.14, -0.001}
    \item \textbf{String (str)}: \textcolor{darkgray}{ข้อความหรืออักขระ เช่น "Hello", "123"}
    \item \textbf{Boolean (bool)}: \textcolor{darkgray}{ค่าจริงหรือเท็จ เช่น True, False}
    \item \textbf{List (list)}: \textcolor{darkgray}{คอลเลกชันของข้อมูลที่สามารถเปลี่ยนแปลงได้ เช่น [1, 2, 3], ["apple", "banana"]}
    \item \textbf{Tuple (tuple)}: \textcolor{darkgray}{คอลเลกชันของข้อมูลที่ไม่สามารถเปลี่ยนแปลงได้ เช่น (1, 2, 3), ("apple", "banana")}
    \item \textbf{Dictionary (dict)}: \textcolor{darkgray}{คอลเลกชันของคู่คีย์-ค่า เช่น \{"name": "John", "age": 30\}}
\end{itemize}

\begin{codebox}{(ตัวอย่าง) ชนิดข้อมูลใน Python}
\begin{lstlisting}[style=python]
# Integer
x = 10

# Float
y = 3.14

# String
name = "First"

# Boolean
is_active = True

# List
scores = [85, 90, 78, 92]

# Tuple
coordinates = (10.5, 20.3)

# Dictionary
person = {"name": "John", "age": 30}

print(x, y, name, is_active, scores, coordinates, person)
\end{lstlisting}
\end{codebox}

\begin{resultbox}{ผลลัพธ์}
\begin{verbatim}
10 3.14 First True [85, 90, 78, 92] (10.5, 20.3) {'name': 'John', 'age': 30}
\end{verbatim}
\end{resultbox}

\textlight{ในแต่ละชนิดข้อมูลจะมี Utility function (ฟังก์ชันที่ใช้ในการจัดการข้อมูล) ที่ช่วยให้สามารถทำงานกับข้อมูลได้ง่ายขึ้น}

\section{คำสั่งแบบมีเงื่อนไข (Conditional Statements)}

\hspace{1cm}\textlight{คำสั่งแบบมีเงื่อนไข (Conditional Statements) เป็นคำสั่งที่ใช้ในการตัดสินใจว่าควรทำอะไรต่อไปในโปรแกรมตามเงื่อนไขที่กำหนด คำสั่งเหล่านี้ช่วยให้โปรแกรมสามารถทำงานได้อย่างยืดหยุ่นและตอบสนองต่อสถานการณ์ต่างๆ (โค้ดด้านล่าง)}

\subsection{การใช้คำสั่ง if เบื้องต้น}

\hspace{1cm}\textlight{คำสั่ง if ใช้เพื่อตรวจสอบเงื่อนไข ถ้าเงื่อนไขเป็นจริง (True) จะทำการดำเนินการตามที่กำหนดไว้ในบล็อกของ if ดังกล่าว}\\
\textbf{สิ่งที่ต้องการ: ฉันต้องการให้แสดงข้อความ `Hello` หากตัวแปร `name` มีค่าเป็น `KMUTNB`}

\begin{codebox}{(ตัวอย่าง) การใช้ if ใน Python}
\begin{lstlisting}[style=python]
name = "KMUTNB"

if name == "KMUTNB":
    print("Hello")
\end{lstlisting}
\end{codebox}

\begin{answerbox}{ไขข้อสงสัย!}
\textlight{\textbf{==} หมายความว่า เท่ากับ (Equal) ใช้เพื่อตรวจสอบว่าค่าของตัวแปร `name` เท่ากับ "KMUTNB" หรือไม่}
\end{answerbox}

\begin{figure}[H]
\centering
\begin{tikzpicture}[node distance=2cm]

% Nodes (if statement)
\node (start) [startstop] {Start};
\node (name) [process, below of=start] {name = "KMUTNB"};

\node (decision) [decision, below of=name, yshift=-1.5cm] {name == "KMUTNB"};

\node (output) [io, right of=decision, xshift=3cm] {"Hello"};

\node (stop) [startstop, below of=decision, yshift=-1.8cm] {Stop};

% Arrows (if statement)
\draw [arrow] (start) -- (name);
\draw [arrow] (name) -- (decision);
\draw [arrow] (decision) -- node[anchor=west, xshift=-0.8cm, yshift=0.4cm] {True} (output);
\draw [arrow] (output) |- (stop);
\draw [arrow] (decision) -- node[anchor=east, yshift=0.3cm] {False} (stop);

\end{tikzpicture}
\caption{ตัวอย่างโฟลวชาร์ตเงื่อนไข if}
\end{figure}

\subsection{การใช้คำสั่ง if-else เบื้องต้น}
\hspace{1cm}\textlight{คำสั่ง if-else ใช้เพื่อตรวจสอบเงื่อนไข ถ้าเงื่อนไขเป็นจริง (True) จะทำการดำเนินการตามที่กำหนดไว้ในบล็อกของ if แต่ถ้าเงื่อนไขเป็นเท็จ (False) จะทำการดำเนินการตามที่กำหนดไว้ในบล็อกของ else}\\[0.5cm]
\textbf{สิ่งที่ต้องการ: ฉันต้องการให้แสดงข้อความ `Hello` หากตัวแปร `name` มีค่าเป็น `KMUTNB` และแสดงข้อความ `Goodbye` หากไม่ตรงกับเงื่อนไข ใดๆ}

\begin{codebox}{(ตัวอย่าง) การใช้ if-else ใน Python}
\begin{lstlisting}[style=python]
name = "KMITL"

if name == "KMUTNB":
    print("Hello")
else:
    print("Goodbye")
\end{lstlisting}
\end{codebox}

\begin{figure}[H]
\centering
\begin{tikzpicture}[node distance=2cm]

% Nodes (if-else statement)
\node (start) [startstop] {Start};
\node (name) [process, below of=start] {name = "KMITL"};

\node (decision) [decision, below of=name, yshift=-1.5cm] {name == "KMUTNB"};

\node (trueoutput) [io, right of=decision, xshift=3cm] {"Hello"};
\node (falseoutput) [io, below of=decision, yshift=-2cm] {"Goodbye"};

\node (stop) [startstop, below of=falseoutput] {Stop};

% Arrows (if-else statement)
\draw [arrow] (start) -- (name);
\draw [arrow] (name) -- (decision);
\draw [arrow] (decision) -- node[anchor=west, xshift=-0.8cm, yshift=0.4cm] {True} (trueoutput);
\draw [arrow] (trueoutput) |- (stop);
\draw [arrow] (decision) -- node[anchor=east, yshift=0.3cm] {False} (falseoutput);
\draw [arrow] (falseoutput) -- (stop);

\end{tikzpicture}
\caption{ตัวอย่างโฟลวชาร์ตเงื่อนไข if-else}
\end{figure}

\subsection{การใช้คำสั่ง if-elif-else เบื้องต้น}
\hspace{1cm}\textlight{คำสั่ง if-elif-else ใช้เพื่อตรวจสอบหลายเงื่อนไข ถ้าเงื่อนไขแรกเป็นจริง (True) จะทำการดำเนินการตามที่กำหนดไว้ในบล็อกของ if แต่ถ้าเงื่อนไขแรกเป็นเท็จ (False) จะตรวจสอบเงื่อนไขถัดไป (elif) และถ้าไม่มีเงื่อนไขใดเป็นจริง จะทำการดำเนินการตามที่กำหนดไว้ในบล็อกของ else}\\[0.5cm]
\textbf{สิ่งที่ต้องการ: ฉันต้องการให้แสดงข้อความ `Hello` หากตัวแปร `name` มีค่าเป็น `KMUTNB` , แสดงข้อความ `Goodbye` หากตัวแปร `name` มีค่าเป็น `KMITL`, และแสดงข้อความ `Unknown` หากไม่ใช่ทั้งสองกรณี}

\begin{codebox}{(ตัวอย่าง) การใช้ if-elif-else ใน Python}
\begin{lstlisting}[style=python]
name = "KMITL"

if name == "KMUTNB":
    print("Hello")
elif name == "KMITL":
    print("Goodbye")
else:
    print("Unknown")
\end{lstlisting}
\end{codebox}

\begin{figure}[H]
\centering
\begin{tikzpicture}[node distance=2cm]

% Nodes (if-else statement)
\node (start) [startstop] {Start};
\node (name) [process, below of=start] {name = "KMUT"};

\node (kmutnbdecision) [decision, below of=name, yshift=-1.5cm] {name == "KMUTNB"};
\node (kmutnbtrue) [io, below of=kmutnbdecision, yshift=-2cm] {"Hello"};

\node (kmitldecision) [decision, right of=kmutnbdecision, xshift=4cm] {name == "KMITL"};
\node (kmitltrue) [io, below of=kmitldecision, yshift=-2cm] {"Goodbye"};

\node (unknown) [io, right of=kmitltrue, xshift=3cm] {"Unknown"};

\node (stop) [startstop, below of=kmutnbtrue] {Stop};

% Arrows (if-else statement)
\draw [arrow] (start) -- (name);
\draw [arrow] (name) -- (kmutnbdecision);

\draw [arrow] (kmutnbdecision) -- node[anchor=west, xshift=-0.6cm, yshift=0.4cm] {False} (kmitldecision.west);
\draw [arrow] (kmutnbdecision) -- node[anchor=west, xshift=-1.2cm, yshift=0.4cm] {True} (kmutnbtrue);
\draw [arrow] (kmutnbtrue) -- (stop);

\draw [arrow] (kmitldecision) -- node[anchor=west, xshift=-1.2cm, yshift=0.4cm] {True} (kmitltrue);
\draw [arrow] (kmitldecision.east) -- ++(3cm,0) |- node[anchor=east, xshift=1.3cm, yshift=3.5cm] {False} (unknown.north);
\draw [arrow] (kmitltrue) |- (stop);

\draw [arrow] (unknown) |- (stop);

\end{tikzpicture}
\caption{ตัวอย่างโฟลวชาร์ตเงื่อนไข if-elif-else}
\end{figure}

\subsection{บททดสอบ}

\begin{noticebox}{แนะนำเครื่องหมายดำเนินการใหม่!}

\textlight{เครื่องหมายดำเนินการทางคณิตศาสตร์ (Arithmetic Operators) ที่ใช้ใน Python มีดังนี้:}

\begin{itemize}
    \item \textbf{+} : \textcolor{darkgray}{การบวก (Addition)}
    \item \textbf{-} : \textcolor{darkgray}{การลบ (Subtraction)}
    \item \textbf{*} : \textcolor{darkgray}{การคูณ (Multiplication)}
    \item \textbf{/} : \textcolor{darkgray}{การหาร (Division) - ผลลัพธ์เป็นทศนิยม}
    \item \textbf{//} : \textcolor{darkgray}{การหารปัดเศษลง (Floor Division) - ผลลัพธ์เป็นจำนวนเต็ม}
    \item \textbf{\%} : \textcolor{darkgray}{การหาเศษเหลือจากการหาร (Modulus)}
    \item \textbf{**} : \textcolor{darkgray}{การยกกำลัง (Exponentiation)}
\end{itemize}

\begin{codebox}{(ตัวอย่าง) การใช้ Arithmetic Operators}
\begin{lstlisting}[style=python]
a = 10
b = 3

print("a + b =", a + b)
print("a - b =", a - b)
print("a * b =", a * b)   
print("a / b =", a / b)
\end{lstlisting}
\end{codebox}

\begin{resultbox}{ผลลัพธ์}
\begin{verbatim}
a + b = 13
a - b = 7
a * b = 30
a / b = 3.333333...
\end{verbatim}
\end{resultbox}

\textlight{การใช้งาน Arithmetic Operators กับ Conditional Statements}

\begin{codebox}{(ตัวอย่าง) การใช้ Arithmetic ตรวจสอบเลข}
\begin{lstlisting}[style=python]
number = 10

if number / 2 == 5:
    print(f"{number}, Wow!")
else:
    print(f"{number}, Okay!")
\end{lstlisting}
\end{codebox}

\end{noticebox}

\begin{noticebox}{แนะนำคำสั่งใหม่!}

\textlight{หากต้องการรับค่าจากผู้ใช้ใน Python สามารถใช้คำสั่ง \textbf{input()} ได้ \textcolor{darkgray}{(ค่าชนิดข้อมูลเริ่มต้นของ input() คือ String)} เช่น}

\begin{codebox}{}
\begin{lstlisting}[style=python]
name = input("Enter your name: ")
print("Hello", name)
\end{lstlisting}
\end{codebox}

\begin{resultbox}{ผลลัพธ์}
\begin{verbatim}
Enter your name: <ใส่ชื่อของคุณ>
Hello <ชื่อของคุณ>
\end{verbatim}
\end{resultbox}

\textlight{หากต้องการแปลงค่าที่รับเข้ามาเป็นชนิดข้อมูลอื่น เช่น แปลงเป็นจำนวนเต็ม สามารถใช้คลาส \textbf{int()} ได้ เช่น}

\begin{codebox}{}
\begin{lstlisting}[style=python]
name = int(input("Enter your age: "))
\end{lstlisting}
\end{codebox}

\textlight{และอื่น ๆ เช่น แปลงเป็นจำนวนทศนิยมด้วย \textbf{float()}}

\end{noticebox}

\subsubsection{แบบฝึกหัดการใช้เงื่อนไข}

\textlight{แบบฝึกหัดเพื่อฝึกการใช้คำสั่งเงื่อนไขในการแก้ปัญหาจริง พร้อมทำ \textcolor{magenta}{Flowchart} และ\textcolor{magenta}{โค้ด Python} ตามโจทย์ที่กำหนด}

\begin{exercisebox}{โจทย์ที่ 1: การตรวจสอบเลขคู่คี่}
\textlight{จงเขียนโปรแกรมที่รับตัวเลขจากผู้ใช้ และแสดงผลว่าตัวเลขนั้นเป็นเลขคู่หรือเลขคี่}
\end{exercisebox}

\begin{exercisebox}{โจทย์ที่ 2: การตรวจสอบค่าเฉลี่ย}
\textlight{จงเขียนโปรแกรมที่ค่าคะแนน 3 ส่วนจากผู้ใช้งาน และหาค่าเฉลี่ยของคะแนนนั้น ถ้าค่าเฉลี่ยมากกว่าหรือเท่ากับ 60 ให้แสดงผลว่า "ผ่าน" ถ้าน้อยกว่า 60 ให้แสดงผลว่า "ไม่ผ่าน"}
\end{exercisebox}

\begin{exercisebox}{โจทย์ที่ 3: การตรวจสอบอายุ}
\textlight{จงเขียนโปรแกรมที่รับอายุจากผู้ใช้ และแสดงผลว่า:}
\begin{itemize}
    \item \textlight{อายุต่ำกว่า 13 ปี: เด็ก}
    \item \textlight{อายุ 13-19 ปี: วัยรุ่น}
    \item \textlight{อายุ 20-59 ปี: ผู้ใหญ่}
    \item \textlight{อายุ 60 ปีขึ้นไป: ผู้สูงอายุ}
\end{itemize}
\end{exercisebox}

\section{การวนซ้ำ (Iteration)}
\hspace{1cm}\textlight{การวนซ้ำ (Iteration) เป็นกระบวนการที่ทำให้โปรแกรมสามารถทำงานซ้ำๆ ตามเงื่อนไขที่กำหนด การวนซ้ำช่วยให้สามารถทำงานกับชุดข้อมูลขนาดใหญ่ได้อย่างมีประสิทธิภาพ โดยไม่ต้องเขียนโค้ดซ้ำๆ หลายครั้ง โดยทั่วไปแล้ว การวนซ้ำใน Python มีสองรูปแบบหลักคือ \textcolor{magenta}{\textbf{for}} และ \textcolor{magenta}{\textbf{while}}}

\begin{noticebox}{แนะนำเครื่องหมายเปรียบเทียบ!}

\textlight{เครื่องหมายเปรียบเทียบ (Comparison Operators) ที่ใช้ในการตรวจสอบเงื่อนไข:}

\begin{itemize}
    \item \textbf{==} : \textcolor{darkgray}{เท่ากับ (Equal to)}
    \item \textbf{!=} : \textcolor{darkgray}{ไม่เท่ากับ (Not equal to)}
    \item \textbf{>} : \textcolor{darkgray}{มากกว่า (Greater than)}
    \item \textbf{<} : \textcolor{darkgray}{น้อยกว่า (Less than)}
    \item \textbf{>=} : \textcolor{darkgray}{มากกว่าหรือเท่ากับ (Greater than or equal to)}
    \item \textbf{<=} : \textcolor{darkgray}{น้อยกว่าหรือเท่ากับ (Less than or equal to)}
\end{itemize}

\begin{codebox}{(ตัวอย่าง) การใช้ Comparison Operators}
\begin{lstlisting}[style=python]
age = 20
score = 85

# Comparison Operators
print("age > 18:", age > 18)           # True
print("score <= 90:", score <= 90)     # True
print("age != 25:", age != 25)         # True
print("age == 20:", age == 20)         # True
print("score >= 80:", score >= 80)     # True
print("age < 25:", age < 25)           # True
\end{lstlisting}
\end{codebox}

\begin{resultbox}{ผลลัพธ์}
\begin{verbatim}
age > 18: True
score <= 90: True
age != 25: True
age == 20: True
score >= 80: True
age < 25: True
\end{verbatim}
\end{resultbox}

\end{noticebox}

\begin{noticebox}{แนะนำเครื่องหมายตรรกะ!}

\textlight{เครื่องหมายตรรกะ (Logical Operators) ที่ใช้ในการรวมเงื่อนไขหลายๆ อัน:}

\begin{itemize}
    \item \textbf{and} : \textcolor{darkgray}{และ - ต้องเป็นจริงทั้งสองฝั่ง}
    \item \textbf{or} : \textcolor{darkgray}{หรือ - เป็นจริงฝั่งใดฝั่งหนึ่ง}
    \item \textbf{not} : \textcolor{darkgray}{ไม่ - กลับค่าความจริง หรือ กลับค่าเป็นเท็จ}
    \item \textbf{in} : \textcolor{darkgray}{อยู่ใน - ตรวจสอบว่าข้อมูลอยู่ในชุดข้อมูลหรือไม่}
\end{itemize}

\begin{codebox}{(ตัวอย่าง) การใช้ Logical Operators}
\begin{lstlisting}[style=python]
age = 20
score = 85
name = "Alice"
subjects = ["Math", "Science", "English"]

# Logical Operators
print("age > 18 and score >= 80:", age > 18 and score >= 80)  # True
print("age < 18 or score > 90:", age < 18 or score > 90)      # False
print("not (age < 18):", not (age < 18))                     # True

# In Operator
print("'Math' in subjects:", "Math" in subjects)              # True
print("'Art' in subjects:", "Art" in subjects)                # False
print("'Alice' in name:", "Alice" in name)                   # True
\end{lstlisting}
\end{codebox}

\begin{resultbox}{ผลลัพธ์}
\begin{verbatim}
age > 18 and score >= 80: True
age < 18 or score > 90: False
not (age < 18): True
'Math' in subjects: True
'Art' in subjects: False
'Alice' in name: True
\end{verbatim}
\end{resultbox}

\end{noticebox}

\textlight{\textbf{เครื่องหมาย ตรรกะ หรือ เปรียบเทียบ} ที่เห็นตามเนื้อหาด้านบนทั้งหมด สามารถนำมาประยุกต์ใช้งานกับ \textbf{if-elif-else (Conditional Statements)} ได้เช่นกัน ซึ่งทำให้เราสามารถตรวจสอบเงื่อนไขได้อย่างยืดหยุ่นและมีประสิทธิภาพมากขึ้น}

\vspace{3cm}

\subsection{การวนซ้ำด้วย \textcolor{magenta}{for}}

\hspace{1cm}\textlight{คำสั่ง \textcolor{magenta}{for} ใช้ในการวนซ้ำผ่านชุดข้อมูล เช่น รายการ (List), ทูเพิล (Tuple), หรือช่วงของตัวเลข (Range) โดยจะทำงานตามจำนวนรอบที่กำหนดไว้ในชุดข้อมูลนั้นหรือตามจำนวนของข้อมูล}

\begin{noticebox}{แนะนำคำสั่งใหม่!}
\textlight{\textbf{range(start, stop, step)} เป็นฟังก์ชันที่ใช้สร้างลำดับของตัวเลข โดยมี 1 พารามิเตอร์ที่จำเป็นต้องใส่คือ start ส่วน stop และ step เป็นพารามิเตอร์ที่ไม่จำเป็นต้องใส่ \\[0.5cm] หากต้องการให้มีการหยุดที่ระยะห่างที่กำหนด สามารถใส่ตัวเลขได้ เช่น range(0, 10) ผลลัพธ์จะเป็นตัวเลขตั้งแต่ 0 ถึง 9 \\[0.5cm] แต่ถ้าหากต้องการให้มีการเพิ่มทีละ 2 สามารถใช้ range(0, 10, 2) ได้ ซึ่งจะได้ผลลัพธ์เป็น 0, 2, 4, 6, 8}
\end{noticebox}

\begin{codebox}{(ตัวอย่าง) การวนซ้ำด้วย range(...) กับ start ใน Python}
\begin{lstlisting}[style=python]
for i in range(3):
    print("Iteration", i)
\end{lstlisting}
\end{codebox}

\begin{resultbox}{ผลลัพธ์}
\begin{verbatim}
Iteration 0
Iteration 1
Iteration 2
\end{verbatim}
\end{resultbox}

\vspace{0.5cm}

\begin{codebox}{(ตัวอย่าง) การวนซ้ำด้วย range(...) กับ stop ใน Python}
\begin{lstlisting}[style=python]
for i in range(0, 2):
    print("Iteration with stop", i)
\end{lstlisting}
\end{codebox}

\begin{resultbox}{ผลลัพธ์}
\begin{verbatim}
Iteration with stop 0
Iteration with stop 1
\end{verbatim}
\end{resultbox}

\vspace{5cm}

\begin{codebox}{(ตัวอย่าง) การวนซ้ำด้วย range(...) กับ step ใน Python}
\begin{lstlisting}[style=python]
for i in range(0, 10, 2):
    print("Iteration with stop and step", i)
\end{lstlisting}
\end{codebox}

\begin{resultbox}{ผลลัพธ์}
\begin{verbatim}
Iteration with stop and step 0
Iteration with stop and step 2
Iteration with stop and step 4
Iteration with stop and step 6
Iteration with stop and step 8
\end{verbatim}
\end{resultbox}

\begin{figure}[H]
\centering
\begin{tikzpicture}[node distance=2cm]

% Nodes (For iteration)
\node (start) [startstop] {Start};

\node (range) [decision, below of=start, yshift=-1.5cm] {i in range(0, 10, 2)};
\node (output) [io, below of=range, yshift=-2cm] {Print "Iteration with stop and step", i};

\node (stop) [startstop, below of=output] {Stop};

% Arrows (For itration)

\draw [arrow] (start) -- (range);

\draw [arrow] (range) -- node[anchor=west, xshift=-1.2cm, yshift=0.4cm] {True} (output);

\draw [arrow] (output.west) |- (range.west);

\draw [arrow] (range.east) -- ++(4cm,0) |- node[anchor=east, xshift=1.3cm, yshift=3.5cm] {False} (stop.east);

\end{tikzpicture}
\caption{ตัวอย่างโฟลวชาร์ตการวนซ้ำด้วย range(start, stop, step)}
\end{figure}

\begin{answerbox}{ไขข้อสงสัย!}
\textlight{\textbf{โฟลวชาร์ต (Flowchart)} สามารถวางรูปแบบใดก็ได้ ตามความเหมาะสมของลักษณะการทำงานของโปรแกรม โดยทิศทางของลูกศรจะชี้ไปยังขั้นตอนถัดไปที่ต้องดำเนินการ การใช้โฟลวชาร์ตช่วยให้เข้าใจลำดับการทำงานของโปรแกรมได้ง่ายขึ้น}
\end{answerbox}

\begin{noticebox}{แนะนำคำสั่งใหม่!}
\textlight{\textbf{len(obj)} คือฟังก์ชันที่ใช้ในการหาความยาวของชุดข้อมูล เช่น รายการ (List), สตริง (String), ทูเพิล (Tuple) หรือ ดิชันนารี (Dictionary) เป็นต้น \\[0.2cm] ซึ่งฟังก์ชันดังกล่าวจะคืนค่ากลับมาเป็น \textcolor{magenta}{จำนวนเต็ม (Integer)} ของชุดข้อมูล นั้นๆ ที่คุณใช้ len(...) \\[0.2cm] หากเป็น Dictionary จะนับจำนวนคีย์ (Keys) ที่มีอยู่ใน Dictionary นั้น ๆ}
\end{noticebox}

\begin{codebox}{(ตัวอย่าง) การใช้ฟังก์ชัน len(...) ใน Python}
\begin{lstlisting}[style=python]
names = ["Alice", "Bob", "Charlie"]
print(len(names))
\end{lstlisting}
\end{codebox}

\begin{resultbox}{ผลลัพธ์}
\begin{verbatim}
3
\end{verbatim}
\end{resultbox}

\vspace{0.5cm}

\textlight{\textbf{Index} หรือ ดัชนี เป็นตัวเลขที่ใช้ระบุตำแหน่งของสมาชิกในชุดข้อมูล เช่น รายการ (List) ใน Python โดยเริ่มนับจาก 0 ดังนั้นสมาชิกแรกจะมีดัชนีเป็น 0, สมาชิกที่สองจะมีดัชนีเป็น 1 และต่อไปเรื่อย ๆ}

\begin{figure}[H]
\centering
\begin{tikzpicture}[node distance=3cm]

% Array elements
\node (elem0) [draw, rectangle, minimum width=2.5cm, minimum height=1cm, fill=cyan!30] {Alice};
\node (elem1) [draw, rectangle, minimum width=2.5cm, minimum height=1cm, fill=cyan!30, right of=elem0] {Bob};
\node (elem2) [draw, rectangle, minimum width=2.5cm, minimum height=1cm, fill=cyan!30, right of=elem1] {Charlie};

% Index labels above
\node (idx0) [above of=elem0, yshift=-0.3cm] {\textbf{Index 0}};
\node (idx1) [above of=elem1, yshift=-0.3cm] {\textbf{Index 1}};
\node (idx2) [above of=elem2, yshift=-0.3cm] {\textbf{Index 2}};

% Array name and brackets
\node (arrayname) [left of=elem0, xshift=-0.5cm] {\textbf{names =}};
\node (leftbracket) [left of=elem0, xshift=1cm] {\textbf{[}};
\node (rightbracket) [right of=elem2, xshift=-1cm] {\textbf{]}};

% Length indicator
\node (lengthinfo) [below of=elem1, yshift=1cm] {\textcolor{magenta}{\textbf{len(names) = 3}}};

% Arrows pointing to indices
\draw [->] (idx0) -- (elem0.north);
\draw [->] (idx1) -- (elem1.north);
\draw [->] (idx2) -- (elem2.north);

\end{tikzpicture}
\caption{การแสดงดัชนี (Index) ของรายการ (List) ใน Python}
\end{figure}

\begin{answerbox}{ไขข้อสงสัย!}
\textlight{\textbf{ดัชนี (Index)} ในรายการ (List) เริ่มนับจาก 0 เสมอ ดังนั้นรายการที่มี 3 สมาชิก จะมีดัชนีเป็น 0, 1, และ 2 ตามลำดับ \\[0.3cm] การใช้ \textbf{names[0]} จะได้ค่า "Alice", \textbf{names[1]} จะได้ค่า "Bob", และ \textbf{names[2]} จะได้ค่า "Charlie" \\[0.3cm] ฟังก์ชัน \textbf{len(names)} จะคืนค่าจำนวนสมาชิกทั้งหมดในรายการ ซึ่งในกรณีนี้คือ 3}
\end{answerbox}

\vspace{3cm}

\textlight{การเข้าถึงสมาชิกในรายการ (List) หรือทูเพิล (Tuple) สามารถทำได้โดยใช้ดัชนี (Index) ซึ่งเป็นตัวเลขที่ระบุตำแหน่งของสมาชิกในชุดข้อมูลนั้น ๆ อย่างเช่น \textbf{names[0]} จะเข้าถึงสมาชิกแรกของรายการ names ซึ่งคือ \textbf{"Alice"} หรือการเข้าถึงแบบ Negative Index เช่น \textbf{names[-1]} จะเข้าถึงสมาชิกสุดท้ายของรายการ names ซึ่งคือ \textbf{"Charlie"}}

\begin{codebox}{(ตัวอย่าง) การเข้าถึงสมาชิกในรายการ (List) ด้วยดัชนี (Index) ใน Python}
\begin{lstlisting}[style=python]
names = ["Alice", "Bob", "Charlie"]
print(names[0])
print(names[-1])
print(names[2])
\end{lstlisting}
\end{codebox}

\begin{resultbox}{ผลลัพธ์}
\begin{verbatim}
Alice
Charlie
Charlie
\end{verbatim}
\end{resultbox}

\vspace{0.5cm}

\textlight{การเข้าถึงสมาชิกแบบ Slice (Slicing) ช่วยให้สามารถเข้าถึงสมาชิกในรายการ (List) หรือทูเพิล (Tuple) ได้หลายตัวพร้อมกัน โดยใช้รูปแบบ \textbf{list[start:stop:step]} ซึ่งจะคืนค่าชุดข้อมูลย่อยที่เริ่มจากดัชนี start ถึง stop (ไม่รวม stop) และเพิ่มทีละ step}

\begin{codebox}{(ตัวอย่าง) การวนซ้ำด้วย range(...) กับ len(...) ใน Python}
\begin{lstlisting}[style=python]
names = ["Alice", "Bob", "Charlie"]
print(names[0:2])  # Slicing from index 0 to 1
print(names[1:])   # Slicing from index 1 to the end
print(names[:3])   # Slicing from the start and stop at index 3 (Take only 0, 1, 2)
print(names[::2])  # Slicing with step 2
\end{lstlisting}
\end{codebox}

\begin{resultbox}{ผลลัพธ์}
\begin{verbatim}
['Alice', 'Bob']
['Bob', 'Charlie']
['Alice', 'Bob', 'Charlie']
['Alice', 'Charlie']
\end{verbatim}
\end{resultbox}

\vspace{4cm}

\begin{codebox}{(ตัวอย่าง) การวนซ้ำด้วย for กับ range(...) และ len(...) ใน Python}
\begin{lstlisting}[style=python]
names = ["Alice", "Bob", "Charlie"]
for i in range(len(names)):
    print(names[i])
\end{lstlisting}
\end{codebox}

\begin{resultbox}{ผลลัพธ์}
\begin{verbatim}
Alice
Bob
Charlie
\end{verbatim}
\end{resultbox}

\begin{figure}[H]
\centering
\begin{tikzpicture}[node distance=2cm]

% Nodes (For iteration)
\node (start) [startstop] {Start};

\node (names) [process, below of=start] {names = ["Alice", "Bob", "Charlie"]};

\node (range) [decision, below of=names, yshift=-1.5cm] {i in range(len(names))};
\node (output) [io, below of=range, yshift=-2cm] {Print names[i]};

\node (stop) [startstop, below of=output] {Stop};

% Arrows (For itration)

\draw [arrow] (start) -- (names);

\draw [arrow] (names) -- (range);

\draw [arrow] (range) -- node[anchor=west, xshift=-1.2cm, yshift=0.4cm] {True} (output);

\draw [arrow] (output.west) |- (range.west);

\draw [arrow] (range.east) -- ++(1cm,0) |- node[anchor=east, xshift=1.3cm, yshift=3.5cm] {False} (stop.east);

\end{tikzpicture}
\caption{ตัวอย่างโฟลวชาร์ตการวนซ้ำด้วย for กับ range(...) และ len(...)}
\end{figure}

\vspace{4cm}

\textlight{หากต้องการวนซ้ำผ่าน ตัวแปร ลิสต์ (List) หรือทูเพิล (Tuple) โดยตรง สามารถใช้คำสั่ง \textcolor{magenta}{for} ได้โดยไม่ต้องใช้ \textcolor{gray}{range} เพียงแค่ใช้ Logical Operator อย่าง \textbf{\textcolor{magenta}{in ...}} เพื่อวนซ้ำผ่านสมาชิกของลิสต์หรือทูเพิลนั้น ๆ}

\begin{codebox}{(ตัวอย่าง) การวนซ้ำด้วย for ใน Python}
\begin{lstlisting}[style=python]
names = ["Alice", "Bob", "Charlie"]

for name in names:
    print(name)
\end{lstlisting}
\end{codebox}

\begin{resultbox}{ผลลัพธ์}
\begin{verbatim}
Alice
Bob
Charlie
\end{verbatim}
\end{resultbox}

\vspace{1.5cm}

\begin{figure}[H]
\centering
\begin{tikzpicture}[node distance=2cm]

% Nodes (For iteration)
\node (start) [startstop] {Start};

\node (names) [process, below of=start] {names = ["Alice", "Bob", "Charlie"]};

\node (range) [decision, below of=names, yshift=-1.5cm] {name in names};
\node (output) [io, below of=range, yshift=-2cm] {Print name};

\node (stop) [startstop, below of=output] {Stop};

% Arrows (For itration)

\draw [arrow] (start) -- (names);

\draw [arrow] (names) -- (range);

\draw [arrow] (range) -- node[anchor=west, xshift=-1.2cm, yshift=0.4cm] {True} (output);

\draw [arrow] (output.west) |- (range.west);

\draw [arrow] (range.east) -- ++(1cm,0) |- node[anchor=east, xshift=1.3cm, yshift=3.5cm] {False} (stop.east);

\end{tikzpicture}
\caption{ตัวอย่างโฟลวชาร์ตการวนซ้ำด้วย for กับ ลิสต์ ตรงๆ}
\end{figure}

\subsubsection{การประยุกต์ใช้ร่วมกับ Conditional Statements}

\textlight{การใช้คำสั่ง \textbf{\textcolor{magenta}{for}} ร่วมกับ \textbf{\textcolor{magenta}{if}} ช่วยให้สามารถกรองข้อมูลหรือทำงานกับข้อมูลที่ตรงตามเงื่อนไขได้ เช่น การหาค่าที่ตรงตามเงื่อนไขในลิสต์ ซึ่งมีประโยชน์มาก ในการจัดการชุดข้อมูลที่มีจำนวนมาก \textbf{(ตัวอย่างด้านล่าง)}}

\begin{noticebox}{แนะนำการประยุกต์ใช้เครื่องหมายดำเนินการ!}

\textlight{หากมีตัวแปร \textbf{X = 10} เมื่อเราใช้ \textbf{X += 5} จะเท่ากับ \textbf{X = X + 5} ซึ่งจะทำให้ \textbf{X} มีค่าเป็น \textbf{15} นั่นหมายความว่าเครื่องหมาย \textbf{+=} เป็นการย่อรูปของการบวกและกำหนดค่าใหม่ให้กับตัวแปร ซึ่งจะนำค่าเดิมของตัวแปรมาบวกกับค่าที่ระบุ และเก็บผลลัพธ์ไว้ในตัวแปรนั้น \textbf{(หากตัวแปรนั้นมีค่าเดิมอยู่แล้ว)}}\\[0.5cm]
\textlight{นอกจากการใช้ \textbf{+=} ยังมีเครื่องหมายดำเนินการอื่น ๆ ที่สามารถใช้ได้เช่นกัน เช่น}

\begin{itemize}
    \item \textbf{-=} : \textcolor{darkgray}{การลบ (Subtraction)}
    \item \textbf{*=} : \textcolor{darkgray}{การคูณ (Multiplication)}
    \item \textbf{/=} : \textcolor{darkgray}{การหาร (Division) - ผลลัพธ์เป็นทศนิยม}
    \item \textbf{//=} : \textcolor{darkgray}{การหารปัดเศษลง (Floor Division) - ผลลัพธ์เป็นจำนวนเต็ม}
    \item \textbf{\%=} : \textcolor{darkgray}{การหาเศษเหลือจากการหาร (Modulus)}
    \item \textbf{**=} : \textcolor{darkgray}{การยกกำลัง (Exponentiation)}
\end{itemize}

\end{noticebox}

\begin{codebox}{(ตัวอย่าง) การวนซ้ำพร้อมประยุกต์ใช้เงื่อนไข ใน Python}
\begin{lstlisting}[style=python]
aggregated = 0
numbers = [39, 89, 72, 27, 15]

for number in numbers:
    if number > 50:
        print("That's a big number:", number)

    aggregated += number

print("Total sum:", aggregated)
\end{lstlisting}
\end{codebox}

\begin{resultbox}{ผลลัพธ์}
\begin{verbatim}
That 's a big number: 89
That 's a big number: 72
Total sum: 241
\end{verbatim}
\end{resultbox}

\begin{figure}[H]
\centering
\begin{tikzpicture}[node distance=2cm]

% Nodes (For iteration)
\node (start) [startstop] {Start};

\node (aggregated) [process, below of=start] {aggregated = 0};
\node (numbers) [process, below of=aggregated] {numbers = [39, 89, 72, 27, 15]};

\node (fornumbers) [decision, below of=numbers, yshift=-1.5cm] {number in numbers};
\node (ifnumber) [decision, below of=fornumbers, yshift=-2.5cm] {number > 50};

\node (output) [io, below of=ifnumber, yshift=-1cm] {Print "That's a big number:", number};

\node (addnumber) [process, below of=output] {aggregated += number};

\node (stop) [startstop, below of=addnumber] {Stop};

% Arrows (For itration)

\draw [arrow] (start) -- (aggregated);

\draw [arrow] (aggregated) -- (numbers);

\draw [arrow] (numbers) -- (fornumbers);

\draw [arrow] (fornumbers) -- node[anchor=west, xshift=-1.2cm, yshift=0.4cm] {True} (ifnumber);

\draw [arrow] (ifnumber) -- node[anchor=west, xshift=-1.2cm, yshift=0.4cm] {True} (output);

\draw [arrow] (ifnumber.east) -- ++(4.5cm,0) |- node[anchor=east, xshift=-3.5cm, yshift=5.5cm] {False} (addnumber.east);

\draw [arrow] (output) -- (addnumber);

\draw [arrow] (addnumber.west) -- ++(-4cm,0) |- (fornumbers.west);

\draw [arrow] (fornumbers.east) -- ++(5.5cm,0) |- node[anchor=east, xshift=-4.5cm, yshift=12cm] {False} (stop.east);

\end{tikzpicture}
\caption{ตัวอย่างโฟลวชาร์ตการวนซ้ำด้วย for และเงื่อนไข if ใน Python}
\end{figure}

\vspace{4cm}

\subsubsection{การวนซ้ำซ้อน (Nested Loops)}

\hspace{1cm}\textlight{การวนซ้ำซ้อน (Nested Loops) คือการใช้คำสั่งวนซ้ำภายในคำสั่งวนซ้ำอีกครั้ง ซึ่งช่วยให้สามารถทำงานกับชุดข้อมูลที่มีโครงสร้างซับซ้อนได้ เช่น การทำงานกับตารางหรือเมทริกซ์ที่มีหลายมิติ หรือ การเรียงตัวเลขจากเล็กไปใหญ่ในลิสต์ เป็นต้น ฯลฯ}

\begin{noticebox}{แนะนำการใช้งาน Placeholder!}

\textlight{การใช้ Placeholder ใน Python ช่วยให้สามารถจัดรูปแบบข้อความได้อย่างยืดหยุ่น โดยใช้สัญลักษณ์ \textbf{\%} เพื่อแทนที่ค่าตัวแปรในสตริง เช่น}

\begin{codebox}{}
\begin{lstlisting}[style=python]
print("%s x %s = %s" % (2, 2, 4))
\end{lstlisting}
\end{codebox}

\textlight{\%s คือ Placeholder ที่ใช้แทนค่าตัวแปรในสตริง ซึ่งจะถูกแทนที่ด้วยค่าที่ระบุใน Tuple หลังจาก \% เช่น \\ ในตัวอย่างนี้ \textbf{2 x 2 = 4} จะถูกพิมพ์ออกมา (\%s อันแรกแทนที่ด้วย 2, อันที่สองแทนที่ด้วย 2, และอันสุดท้ายแทนที่ด้วย 4) \\[0.5cm] นอกจากนี้ยังสามารถใช้ Placeholder อื่น ๆ เช่น \%d สำหรับจำนวนเต็ม (Integer), \%f สำหรับจำนวนทศนิยม (Float) เป็นต้น ฯลฯ}

\end{noticebox}

\begin{codebox}{(ตัวอย่าง) การวนซ้ำซ้อนเพื่อแสดงตารางสูตรคูณ 1 ถึง 12 ใน Python}
\begin{lstlisting}[style=python]
for i in range(1, 13):
    for j in range(1, 13):
        print("%s x %s = %s" % (i, j, i * j))
\end{lstlisting}
\end{codebox}

\begin{resultbox}{ผลลัพธ์}
\begin{verbatim}
1 x 1 = 1
1 x 2 = 2
1 x 3 = 3
1 x 4 = 4
1 x 5 = 5
1 x 6 = 6
1 x 7 = 7
1 x 8 = 8
1 x 9 = 9
1 x 10 = 10
1 x 11 = 11
1 x 12 = 12
...
12 x 12 = 144
\end{verbatim}
\end{resultbox}

\vspace{4cm}

\begin{figure}[H]
\centering
\begin{tikzpicture}[node distance=2cm]

% Nodes (For iteration)
\node (start) [startstop] {Start};

\node (irange) [decision, below of=start, yshift=-1.5cm] {i in range(1, 13)};
\node (jrange) [decision, below of=irange, yshift=-2.5cm] {j in range(1, 13)};

\node (output) [io, below of=jrange, yshift=-1.2cm] {Print "\%s x \%s = \%s" \% (i, j, i * j)};

\node (stop) [startstop, below of=output] {Stop};

% Arrows (For itration)

\draw [arrow] (start) -- (irange);

\draw [arrow] (irange) -- node[anchor=west, xshift=-1.2cm, yshift=0.4cm] {True} (jrange);

\draw [arrow] (jrange) -- node[anchor=west, xshift=-1.2cm, yshift=0.4cm] {True} (output);

\draw [arrow] (jrange.west) -- ++(-2cm,0) |- node[anchor=east, xshift=2cm, yshift=-4cm] {False} (irange.west);

\draw [arrow] (output.west) -- ++(-1cm,0) |- (irange.west);

\draw [arrow] (irange.east) -- ++(4cm,0) |- node[anchor=east, xshift=-3cm, yshift=10.2cm] {False} (stop.east);

\end{tikzpicture}
\caption{ตัวอย่างโฟลวชาร์ตการวนซ้ำด้วย for และเงื่อนไข if ใน Python}
\end{figure}

\vspace{10cm}

\subsubsection{แบบฝึกหัดการใช้การวนซ้ำ}
\textlight{แบบฝึกหัดเพื่อฝึกการใช้คำสั่งวนซ้ำในการแก้ปัญหาจริง พร้อมทำ \textcolor{magenta}{Flowchart} และ\textcolor{magenta}{โค้ด Python} ตามโจทย์ที่กำหนด}

\begin{exercisebox}{โจทย์ที่ 1: การคัดกรองเลขที่ไม่ซ้ำกัน}
\textlight{จงเขียนโปรแกรมที่วนซ้ำ \textbf{ชุดข้อมูลตัวเลข (List of Numbers)} ที่มีการซ้ำกัน เช่น \textbf{[1, 2, 3, 4, 5, 1, 2, 3]} และสร้างชุดข้อมูลใหม่ที่มีตัวเลขที่ไม่ซ้ำกัน เช่น \textbf{[1, 2, 3, 4, 5]} โดยใช้คำสั่งวนซ้ำ \textbf{\textcolor{magenta}{for}} และเงื่อนไข \textbf{\textcolor{magenta}{if}} เพื่อกรองข้อมูล}
\end{exercisebox}

\begin{exercisebox}{โจทย์ที่ 2: หาตำแหน่งของตัวเลข 2 ตัวที่เป็นรากของผลรวม}
\textlight{จงเขียนโปรแกรมที่วนซ้ำ \textbf{ชุดข้อมูลตัวเลข (List of Numbers)} และ หาตำแหน่ง (Index) ของตัวเลข 2 ตัวแรก ที่มีผลรวมเท่ากับ 10 เช่น \textbf{[1, 2, 3, 7, 8]} จะได้ตำแหน่งของตัวเลข 2 ตัวที่เป็นรากของผลรวมคือ (3, 7) เป็นต้น โดยใช้คำสั่งวนซ้ำ \textbf{\textcolor{magenta}{for}} และเงื่อนไข \textbf{\textcolor{magenta}{if}}}
\end{exercisebox}

\end{document}