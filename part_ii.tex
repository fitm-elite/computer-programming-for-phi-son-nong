\documentclass[12pt,a4paper]{article}

% Page setup
\usepackage[a4paper,top=1in,bottom=1in,left=0.5in,right=0.5in]{geometry}
\usepackage{setspace}
\onehalfspacing

\usepackage{etoolbox}
\usepackage[nonumberlist]{glossaries}
\usepackage{hyperref}
\usepackage{graphicx}
\usepackage{float}
\usepackage{xcolor}
\usepackage{colortbl}

% Table packages
\usepackage{array}
\usepackage{booktabs}
\usepackage{longtable}
\usepackage{tabularx}
\usepackage{multirow}
\usepackage{multicol}

% List formatting
\usepackage{enumitem}

% Dotted lines package
\usepackage{dashrule}

% Code listings package for syntax highlighting
\usepackage{listings}
\usepackage{xcolor}
\usepackage{tcolorbox}
\tcbuselibrary{listings,skins}

% Thai language support
\usepackage{xunicode}
\usepackage{xltxtra}

% Optimized fonts for Thai language - using Medium weight
\setmainfont[
    Path=./font/Sarabun/,
    UprightFont=*-Regular,
    BoldFont=*-Medium,
    ItalicFont=*-Italic,
    BoldItalicFont=*-MediumItalic,
    Scale=0.90,
    Ligatures=TeX,
    WordSpace=1.2,
    PunctuationSpace=1.1
]{Sarabun}

\newfontfamily\thaifont[
    Path=./font/Sarabun/,
    UprightFont=*-Regular,
    BoldFont=*-Medium,
    ItalicFont=*-Italic,
    BoldItalicFont=*-MediumItalic,
    Scale=0.90,
    Ligatures=TeX,
    WordSpace=1.2,
    PunctuationSpace=1.1,
    Script=Thai,
    Language=Thai
]{Sarabun}

% Light weight font family optimized for Thai
\newfontfamily\thailightfont[
    Path=./font/Sarabun/,
    UprightFont=*-Light,
    BoldFont=*-Medium,
    ItalicFont=*-LightItalic,
    BoldItalicFont=*-MediumItalic,
    Scale=0.90,
    Ligatures=TeX,
    WordSpace=1.2,
    PunctuationSpace=1.1,
    Script=Thai,
    Language=Thai
]{Sarabun}

% Font commands for easy switching
\newcommand{\textlight}[1]{{\thailightfont #1}}
\newcommand{\thailight}{\thailightfont}

% Thai typography optimization commands
\newcommand{\optimalthai}[1]{{\thaifont #1}}

% Monospace font for Thai in code blocks
\newfontfamily\thaifonttt[
    Path=./font/Sarabun/,
    UprightFont=*-Regular,
    BoldFont=*-Medium,
    Scale=0.9,
    Ligatures=TeX,
    Script=Thai,
    Language=Thai
]{Sarabun}

% Additional packages for better Thai support
\usepackage{polyglossia}
\setdefaultlanguage{thai}
\setotherlanguage{english}

% Thai line breaking settings - optimized
\XeTeXlinebreaklocale "th"
\XeTeXlinebreakskip = 0pt plus 2pt minus 1pt

% Thai typography enhancements
\frenchspacing 
\tolerance=1000
\emergencystretch=3em

% Thai punctuation spacing
\XeTeXinterchartokenstate = 1

% Better Thai paragraph spacing
\setlength{\parskip}{0.5\baselineskip plus 0.2\baselineskip minus 0.1\baselineskip}
\setlength{\parindent}{2em}

% Custom section numbering with period and reduced spacing
\renewcommand{\thesection}{\arabic{section}.}
\renewcommand{\thesubsection}{\arabic{section}.\arabic{subsection}.}
\renewcommand{\thesubsubsection}{\arabic{section}.\arabic{subsection}.\arabic{subsubsection}}

% Custom dotted rule command
\newcommand{\dotrule}[1]{\hdashrule{#1}{0.6pt}{1pt}}

% Flowchart and diagram packages
\usepackage{tikz}
\usepackage{tikz-qtree}
\usetikzlibrary{shapes.geometric, arrows, positioning, fit, calc}

% Define flowchart styles
\tikzstyle{startstop} = [rectangle, rounded corners, minimum width=3cm, minimum height=1cm, text centered, draw=black, fill=red!30]
\tikzstyle{process} = [rectangle, minimum width=3cm, minimum height=1cm, text centered, draw=black, fill=blue!30]
\tikzstyle{decision} = [diamond, minimum width=1.5cm, minimum height=1cm, text centered, draw=black, fill=green!30]
\tikzstyle{io} = [trapezium, trapezium left angle=70, trapezium right angle=110, minimum width=2cm, minimum height=1cm, text centered, draw=black, fill=yellow!30]
\tikzstyle{arrow} = [thick,->,>=stealth]

% Code styling configuration
\definecolor{codegreen}{rgb}{0,0.6,0}
\definecolor{codegray}{rgb}{0.5,0.5,0.5}
\definecolor{codepurple}{rgb}{0.58,0,0.82}
\definecolor{backcolour}{rgb}{1,1,1}

% Monospace font for code
\newfontfamily\codefont{Menlo}[
    Scale=0.85
]

% Python code style
\lstdefinestyle{python}{
    backgroundcolor=\color{backcolour},   
    commentstyle=\color{codegreen},
    keywordstyle=\color{magenta},
    numberstyle=\tiny\color{codegray},
    stringstyle=\color{codepurple},
    basicstyle=\codefont\footnotesize,
    breakatwhitespace=false,         
    breaklines=true,                 
    captionpos=b,                    
    keepspaces=true,                 
    numbers=left,                    
    numbersep=10pt,            
    xleftmargin=10pt,
    showspaces=false,                
    showstringspaces=false,
    showtabs=false,                  
    tabsize=2,
    columns=fixed,
    basewidth=0.6em,
    language=Python
}

% Custom code box using tcolorbox
\newtcolorbox{codebox}[2][]{
    colback=white!5!white,
    colframe=gray!75!black,
    fonttitle=\bfseries,
    title=#2,
    top=0pt,
    bottom=0pt,
    boxsep=0pt,
    toptitle=0pt,
    bottomtitle=0pt,
    #1
}

% Custom result box for output display
\newtcolorbox{resultbox}[2][]{
    colback=white!5!white,
    colframe=green!75!black,
    fonttitle=\bfseries,
    title=#2,
    top=3pt,
    bottom=3pt,
    left=4pt,
    #1
}

% Custom exercise box for exercises
\newtcolorbox{exercisebox}[2][]{
    colback=white!5!white,
    colframe=gray!75!black,
    fonttitle=\bfseries,
    title=#2,
    #1
}

% Custom notice box for important messages
\newtcolorbox{noticebox}[2][]{
    colback=yellow!5!white,
    colframe=magenta!90!black,
    fonttitle=\bfseries,
    title=#2,
    left=6pt,
    right=6pt,
    top=6pt,
    bottom=6pt,
    boxrule=2pt,
    arc=3pt,
    auto outer arc,
    #1
}

% Custom notice box for important messages
\newtcolorbox{answerbox}[2][]{
    colback=yellow!5!white,
    colframe=magenta!50!black,
    fonttitle=\bfseries,
    title=#2,
    left=6pt,
    right=6pt,
    top=6pt,
    bottom=6pt,
    boxrule=2pt,
    arc=3pt,
    auto outer arc,
    #1
}

\begin{document}

% Title Page
\begin{titlepage}
    \centering
    \vspace*{2cm}
    
    {\Huge\bfseries {คอมพิวเตอร์โปรแกรมมิ่ง (โครงการพี่ติวน้อง)}\par}
    {\textlight{เพื่อส่งเสริมการมีส่วนร่วมของรุ่นพี่ในการถ่ายทอดความรู้แก้รุ่นน้อง เพื่อเสริมความเข้าใจในรายวิชา}\par}
    \vspace{2cm}

    {\textlight{\textcolor{darkgray}{*ยงยุทธ ชวนขุนทด, เมธัส ทองจันทร์, ศุภกร จิรศิริวรกุล, มชัญชยา ประยูรมณีรัตน์}}\par}
    {\normalfont ประจำวันที่ 31 กรกฎาคม พ.ศ. 2568\par}
    
    \vfill
    
    {\thailight ภาควิชาเทคโนโลยีสารสนเทศ\par}
    {\thailight มหาวิทยาลัยเทคโนโลยีพระจอมเกล้าพระนครเหนือ วิทยาเขตปราจีนบุรี\par}
\end{titlepage}

% Table of Contents
\tableofcontents
\newpage

\section{รายการและเมธอดสำหรับชนิดข้อมูล (Lists and Data Type Methods)}

\hspace{1cm}\textlight{รายการ (List) เป็นโครงสร้างข้อมูลพื้นฐานใน Python ที่ใช้เก็บข้อมูลหลายค่าไว้ในตัวแปรเดียว คิดเหมือนกับลิสต์รายการสิ่งของที่เราใช้ในชีวิตประจำวัน เช่น รายการซื้อของ รายชื่อเพื่อน หรือรายการคะแนนสอบ ข้อดีของ List คือสามารถเปลี่ยนแปลงข้อมูลได้หลังจากสร้างแล้ว (Mutable) และสามารถเก็บข้อมูลหลายประเภทไว้ด้วยกันได้ เช่น ตัวเลข ข้อความ และค่าความจริง การเรียนรู้การใช้งาน List และเมธอดต่างๆ จะช่วยให้นักศึกษาสามารถจัดการข้อมูลได้อย่างมีประสิทธิภาพ}

\subsection{การสร้างและเข้าถึงรายการ (Creating and Accessing Lists)}

\hspace{1cm}\textlight{การสร้าง List ใน Python ง่ายมาก เพียงใช้เครื่องหมายวงเล็บเหลี่ยม [ ] และคั่นข้อมูลแต่ละตัวด้วยเครื่องหมายจุลภาค การเข้าถึงข้อมูลใน List ใช้หมายเลขดัชนี (Index) ซึ่งเริ่มนับจาก 0 สำหรับตำแหน่งแรก สิ่งที่ควรจำคือ Python รองรับการนับดัชนีแบบย้อนกลับด้วย โดยใช้เลขลบ เช่น -1 สำหรับตำแหน่งสุดท้าย นอกจากนี้ยังสามารถเลือกข้อมูลหลายตำแหน่งพร้อมกันได้ด้วยเทคนิค Slicing}

\begin{codebox}{(ตัวอย่าง) การสร้างและเข้าถึงรายการใน Python}
\begin{lstlisting}[style=python]
# Creating lists
fruits = ["apple", "banana", "orange", "grape"]
numbers = [1, 2, 3, 4, 5]
mixed = ["hello", 42, 3.14, True]

# Accessing list elements
print("First fruit:", fruits[0])
print("Last fruit:", fruits[-1])
print("Second number:", numbers[1])

# Slicing access
print("First 3 fruits:", fruits[0:3])
print("Last 2 fruits:", fruits[-2:])
\end{lstlisting}
\end{codebox}

\begin{resultbox}{ผลลัพธ์}
\begin{verbatim}
First fruit: apple
Last fruit: grape
Second number: 2
First 3 fruits: ['apple', 'banana', 'orange']
Last 2 fruits: ['orange', 'grape']
\end{verbatim}
\end{resultbox}

\begin{figure}[H]
\centering
\begin{tikzpicture}[node distance=2.5cm]

% Array elements
\node (elem0) [draw, rectangle, minimum width=2cm, minimum height=1cm, fill=cyan!30] {apple};
\node (elem1) [draw, rectangle, minimum width=2cm, minimum height=1cm, fill=cyan!30, right of=elem0] {banana};
\node (elem2) [draw, rectangle, minimum width=2cm, minimum height=1cm, fill=cyan!30, right of=elem1] {orange};
\node (elem3) [draw, rectangle, minimum width=2cm, minimum height=1cm, fill=cyan!30, right of=elem2] {grape};

% Index labels above
\node (idx0) [above of=elem0, yshift=-0.3cm] {\textbf{[0]}};
\node (idx1) [above of=elem1, yshift=-0.3cm] {\textbf{[1]}};
\node (idx2) [above of=elem2, yshift=-0.3cm] {\textbf{[2]}};
\node (idx3) [above of=elem3, yshift=-0.3cm] {\textbf{[3]}};

% Negative index labels below
\node (nidx3) [below of=elem0, yshift=0.3cm] {\textbf{[-4]}};
\node (nidx2) [below of=elem1, yshift=0.3cm] {\textbf{[-3]}};
\node (nidx1) [below of=elem2, yshift=0.3cm] {\textbf{[-2]}};
\node (nidx0) [below of=elem3, yshift=0.3cm] {\textbf{[-1]}};

% Array name
\node (arrayname) [left of=elem0, xshift=-1cm] {\textbf{fruits =}};

\end{tikzpicture}
\caption{การแสดงดัชนีของรายการ (Positive และ Negative Index)}
\end{figure}

\subsection{เมธอดสำหรับการจัดการรายการ (List Methods)}

\hspace{1cm}\textlight{เมธอด (Methods) คือฟังก์ชันพิเศษที่ผูกติดกับ List และช่วยให้เราจัดการข้อมูลได้สะดวกขึ้น คิดเหมือนกับปุ่มกดต่างๆ บนรีโมทคอนโทรล แต่ละปุ่มมีหน้าที่เฉพาะ เช่น เพิ่มข้อมูล ลบข้อมูล หรือจัดเรียงข้อมูล การเรียนรู้เมธอดเหล่านี้จะช่วยให้นักศึกษาเขียนโปรแกรมได้อย่างมีประสิทธิภาพและลดความซับซ้อนของโค้ด}

\begin{noticebox}{เมธอดสำคัญของรายการ (List Methods)}

\textlight{เมธอดที่ใช้บ่อยในการจัดการรายการ:}

\begin{itemize}
    \item \textbf{append(item)}: \textcolor{darkgray}{เพิ่มสมาชิกใหม่ที่ท้ายรายการ}
    \item \textbf{insert(index, item)}: \textcolor{darkgray}{เพิ่มสมาชิกใหม่ที่ตำแหน่งที่กำหนด}
    \item \textbf{remove(item)}: \textcolor{darkgray}{ลบสมาชิกที่ระบุออกจากรายการ}
    \item \textbf{pop(index)}: \textcolor{darkgray}{ลบและคืนค่าสมาชิกที่ตำแหน่งที่กำหนด}
    \item \textbf{index(item)}: \textcolor{darkgray}{หาตำแหน่งของสมาชิกที่ระบุ}
    \item \textbf{count(item)}: \textcolor{darkgray}{นับจำนวนสมาชิกที่ระบุ}
    \item \textbf{sort()}: \textcolor{darkgray}{เรียงลำดับรายการ}
    \item \textbf{reverse()}: \textcolor{darkgray}{กลับลำดับรายการ}
    \item \textbf{clear()}: \textcolor{darkgray}{ลบสมาชิกทั้งหมดในรายการ}
\end{itemize}

\end{noticebox}

\begin{codebox}{(ตัวอย่าง) การใช้เมธอด append() และ insert() ใน Python}
\begin{lstlisting}[style=python]
# Create fruit list
fruits = ["apple", "banana"]
print("Original fruits:", fruits)

# Add fruit to the end of the list
fruits.append("orange")
print("After append:", fruits)

# Insert fruit at position 1
fruits.insert(1, "mango")
print("After insert:", fruits)

# Add multiple fruits at once
more_fruits = ["grape", "kiwi"]
fruits.extend(more_fruits)
print("After extend:", fruits)
\end{lstlisting}
\end{codebox}

\begin{resultbox}{ผลลัพธ์}
\begin{verbatim}
Original fruits: ['apple', 'banana']
After append: ['apple', 'banana', 'orange']
After insert: ['apple', 'mango', 'banana', 'orange']
After extend: ['apple', 'mango', 'banana', 'orange', 'grape', 'kiwi']
\end{verbatim}
\end{resultbox}

\vspace{2cm}

\begin{codebox}{(ตัวอย่าง) การใช้เมธอด remove() และ pop() ใน Python}
\begin{lstlisting}[style=python]
# Create number list
numbers = [1, 2, 3, 4, 5, 3, 6]
print("Original numbers:", numbers)

# Remove number 3 (first occurrence)
numbers.remove(3)
print("After remove(3):", numbers)

# Remove element at position 2 and store the removed value
removed_item = numbers.pop(2)
print("After pop(2):", numbers)
print("Removed item:", removed_item)

# Remove last element
last_item = numbers.pop()
print("After pop():", numbers)
print("Last item:", last_item)
\end{lstlisting}
\end{codebox}

\begin{resultbox}{ผลลัพธ์}
\begin{verbatim}
Original numbers: [1, 2, 3, 4, 5, 3, 6]
After remove(3): [1, 2, 4, 5, 3, 6]
After pop(2): [1, 2, 5, 3, 6]
Removed item: 4
After pop(): [1, 2, 5, 3]
Last item: 6
\end{verbatim}
\end{resultbox}

\begin{codebox}{(ตัวอย่าง) การใช้เมธอด sort() และ reverse() ใน Python}
\begin{lstlisting}[style=python]
# Create number list
numbers = [64, 34, 25, 12, 22, 11, 90]
print("Original numbers:", numbers)

# Sort in ascending order
numbers.sort()
print("After sort():", numbers)

# Sort in descending order
numbers.sort(reverse=True)
print("After sort(reverse=True):", numbers)

# Reverse the list
numbers.reverse()
print("After reverse():", numbers)

# Sort words
words = ["python", "java", "c++", "javascript"]
words.sort()
print("Sorted words:", words)
\end{lstlisting}
\end{codebox}

\begin{resultbox}{ผลลัพธ์}
\begin{verbatim}
Original numbers: [64, 34, 25, 12, 22, 11, 90]
After sort(): [11, 12, 22, 25, 34, 64, 90]
After sort(reverse=True): [90, 64, 34, 25, 22, 12, 11]
After reverse(): [11, 12, 22, 25, 34, 64, 90]
Sorted words: ['c++', 'java', 'javascript', 'python']
\end{verbatim}
\end{resultbox}

\vspace{6cm}

\subsection{เมธอดสำหรับชนิดข้อมูลอื่นๆ (Methods for Other Data Types)}

\hspace{1cm}\textlight{เช่นเดียวกับ List สตริง (String) ก็มีเมธอดของตัวเองที่ช่วยจัดการข้อความได้อย่างมีประสิทธิภาพ เมธอดเหล่านี้มีประโยชน์มากในการประมวลผลข้อความ เช่น การทำความสะอาดข้อมูล การค้นหาคำ หรือการแปลงรูปแบบข้อความ ซึ่งเป็นทักษะพื้นฐานที่นักศึกษาจะใช้บ่อยในการเขียนโปรแกรม}

\begin{noticebox}{เมธอดสำคัญของสตริง (String Methods)}

\textlight{เมธอดที่ใช้บ่อยในการจัดการสตริง:}

\begin{itemize}
    \item \textbf{upper()}: \textcolor{darkgray}{แปลงเป็นตัวพิมพ์ใหญ่}
    \item \textbf{lower()}: \textcolor{darkgray}{แปลงเป็นตัวพิมพ์เล็ก}
    \item \textbf{strip()}: \textcolor{darkgray}{ลบช่องว่างหน้าและหลัง}
    \item \textbf{split(separator)}: \textcolor{darkgray}{แยกสตริงตามตัวคั่น}
    \item \textbf{join(list)}: \textcolor{darkgray}{รวมรายการเป็นสตริง}
    \item \textbf{replace(old, new)}: \textcolor{darkgray}{แทนที่ข้อความ}
    \item \textbf{find(substring)}: \textcolor{darkgray}{หาตำแหน่งของข้อความ}
    \item \textbf{startswith(prefix)}: \textcolor{darkgray}{ตรวจสอบว่าเริ่มต้นด้วยข้อความที่ระบุ}
    \item \textbf{endswith(suffix)}: \textcolor{darkgray}{ตรวจสอบว่าลงท้ายด้วยข้อความที่ระบุ}
\end{itemize}

\end{noticebox}

\begin{codebox}{(ตัวอย่าง) การใช้เมธอดของสตริง ใน Python}
\begin{lstlisting}[style=python]
# Using basic string methods
text = "  Hello Python World  "
print("Original:", repr(text))
print("Upper:", text.upper())
print("Lower:", text.lower())
print("Strip:", text.strip())

# Splitting and joining strings
sentence = "Python is awesome"
words = sentence.split()
print("Split:", words)
print("Join:", "-".join(words))

# Replacing and searching
message = "I love Java programming"
new_message = message.replace("Java", "Python")
print("Replace:", new_message)
print("Find Python:", new_message.find("Python"))
print("Starts with 'I':", message.startswith("I"))
\end{lstlisting}
\end{codebox}

\begin{resultbox}{ผลลัพธ์}
\begin{verbatim}
Original: '  Hello Python World  '
Upper:   HELLO PYTHON WORLD  
Lower:   hello python world  
Strip: Hello Python World
Split: ['Python', 'is', 'awesome']
Join: Python-is-awesome
Replace: I love Python programming
Find Python: 7
Starts with 'I': True
\end{verbatim}
\end{resultbox}

\subsection{บททดสอบ}

\begin{exercisebox}{โจทย์ที่ 1: โปรแกรมจัดการรายการอาหาร}
\textlight{จงเขียนโปรแกรมจัดการรายการอาหารโปรดง่ายๆ โดยใช้ List methods}

\textlight{\textbf{ความต้องการ:}}
\begin{itemize}
    \item \textlight{สร้างรายการอาหารเริ่มต้น: ["pizza", "burger", "salad"]}
    \item \textlight{เพิ่มอาหาร "pasta" ลงในรายการ}
    \item \textlight{ลบอาหาร "salad" ออกจากรายการ}
    \item \textlight{แสดงรายการอาหารทั้งหมดเรียงตามตัวอักษร}
    \item \textlight{นับจำนวนอาหารทั้งหมด}
\end{itemize}

\textlight{\textbf{ตัวอย่าง Output ที่ต้องการ:}}
\begin{codebox}{ผลลัพธ์ที่คาดหวัง}
\begin{lstlisting}[style=python]
Initial foods: ['pizza', 'burger', 'salad']
After adding pasta: ['pizza', 'burger', 'salad', 'pasta']
After removing salad: ['pizza', 'burger', 'pasta']
Sorted foods: ['burger', 'pasta', 'pizza']
Total foods: 3
\end{lstlisting}
\end{codebox}
\textlight{พร้อมสร้าง Flowchart แสดงการทำงานของโปรแกรม}
\end{exercisebox}

\begin{exercisebox}{โจทย์ที่ 2: โปรแกรมประมวลผลชื่อ}
\textlight{จงเขียนโปรแกรมประมวลผลชื่อง่ายๆ โดยใช้ String methods}

\textlight{\textbf{ความต้องการ:}}
\begin{itemize}
    \item \textlight{รับชื่อ: "john doe"}
    \item \textlight{แปลงเป็นตัวพิมพ์ใหญ่}
    \item \textlight{นับจำนวนตัวอักษร (ไม่นับช่องว่าง)}
    \item \textlight{แยกชื่อและนามสกุล}
    \item \textlight{สร้างชื่อย่อ (ตัวอักษรแรกของแต่ละคำ)}
\end{itemize}

\textlight{\textbf{ตัวอย่าง Output ที่ต้องการ:}}
\begin{codebox}{ผลลัพธ์ที่คาดหวัง}
\begin{lstlisting}[style=python]
Original name: john doe
Uppercase name: JOHN DOE
Number of characters: 7
First name: john
Last name: doe
Initials: J.D.
\end{lstlisting}
\end{codebox}
\textlight{พร้อมสร้าง Flowchart แสดงการทำงานของโปรแกรม}
\end{exercisebox}

\begin{exercisebox}{โจทย์ที่ 3: โปรแกรมคำนวณคะแนนง่าย}
\textlight{จงเขียนโปรแกรมคำนวณคะแนนง่ายๆ โดยใช้ List methods}

\textlight{\textbf{ความต้องการ:}}
\begin{itemize}
    \item \textlight{มีคะแนนสอบ: [85, 90, 78, 92, 88]}
    \item \textlight{หาคะแนนสูงสุดและต่ำสุด}
    \item \textlight{คำนวณค่าเฉลี่ย}
    \item \textlight{นับจำนวนคะแนนที่มากกว่า 85}
    \item \textlight{เรียงคะแนนจากมากไปน้อย}
\end{itemize}

\textlight{\textbf{ตัวอย่าง Output ที่ต้องการ:}}
\begin{codebox}{ผลลัพธ์ที่คาดหวัง}
\begin{lstlisting}[style=python]
Scores: [85, 90, 78, 92, 88]
Highest score: 92
Lowest score: 78
Average score: 86.6
Scores above 85: 3
Sorted scores: [92, 90, 88, 85, 78]
\end{lstlisting}
\end{codebox}
\textlight{พร้อมสร้าง Flowchart แสดงการทำงานของโปรแกรม}
\end{exercisebox}

\section{ฟังก์ชัน (Functions)}

\hspace{1cm}\textlight{ฟังก์ชัน (Functions) เป็นชุดคำสั่งที่ถูกจัดกลุ่มไว้ด้วยกันเพื่อทำงานเฉพาะอย่าง ฟังก์ชันช่วยให้โค้ดมีความเป็นระเบียบ สามารถนำกลับมาใช้ได้ (Reusable) และง่ายต่อการบำรุงรักษา การเขียนฟังก์ชันที่ดีจะช่วยลดการเขียนโค้ดซ้ำๆ และทำให้โปรแกรมมีประสิทธิภาพมากขึ้น}

\subsection{การสร้างและเรียกใช้ฟังก์ชัน (Creating and Calling Functions)}

\hspace{1cm}\textlight{การสร้างฟังก์ชันใน Python ใช้คำสั่ง \textbf{def} ตามด้วยชื่อฟังก์ชันและพารามิเตอร์ในวงเล็บ การเรียกใช้ฟังก์ชันทำได้โดยเขียนชื่อฟังก์ชันตามด้วยวงเล็บและใส่อาร์กิวเมนต์ (ถ้ามี)}

\begin{noticebox}{โครงสร้างของฟังก์ชัน}

\textlight{โครงสร้างพื้นฐานของฟังก์ชันใน Python:}

\begin{codebox}{}
\begin{lstlisting}[style=python]
def function_name(parameters):
    """Function description (Docstring)"""
    # Function body
    return value  # Optional
\end{lstlisting}
\end{codebox}

\textlight{ส่วนประกอบสำคัญ:}
\begin{itemize}
    \item \textbf{def}: \textcolor{darkgray}{คำสั่งสำหรับประกาศฟังก์ชัน}
    \item \textbf{function\_name}: \textcolor{darkgray}{ชื่อของฟังก์ชัน}
    \item \textbf{parameters}: \textcolor{darkgray}{พารามิเตอร์ที่ฟังก์ชันรับเข้ามา}
    \item \textbf{return}: \textcolor{darkgray}{คำสั่งส่งค่ากลับ (ไม่จำเป็นต้องมี)}
\end{itemize}

\end{noticebox}

\begin{codebox}{(ตัวอย่าง) ฟังก์ชันพื้นฐานใน Python}
\begin{lstlisting}[style=python]
# Simple function without parameters
def say_hello():
    print("Hello, World!")

# Function with parameters
def greet(name):
    print(f"Hello, {name}!")

# Function that returns a value
def add_numbers(a, b):
    result = a + b
    return result

# Calling functions
say_hello()
greet("Alice")
sum_result = add_numbers(5, 3)
print("Sum:", sum_result)
\end{lstlisting}
\end{codebox}

\begin{resultbox}{ผลลัพธ์}
\begin{verbatim}
Hello, World!
Hello, Alice!
Sum: 8
\end{verbatim}
\end{resultbox}

\begin{figure}[H]
\centering
\begin{tikzpicture}[node distance=2cm]

% Function definition flowchart
\node (start) [startstop] {Start};
\node (def) [process, below of=start] {def add\_numbers(a, b)};
\node (calc) [process, below of=def] {result = a + b};
\node (return) [process, below of=calc] {return result};
\node (call) [io, below of=return] {sum\_result = add\_numbers(5, 3)};
\node (print) [io, below of=call] {print("Sum:", sum\_result)};
\node (stop) [startstop, below of=print] {Stop};

\draw [arrow] (start) -- (def);
\draw [arrow] (def) -- (calc);
\draw [arrow] (calc) -- (return);
\draw [arrow] (return) -- (call);
\draw [arrow] (call) -- (print);
\draw [arrow] (print) -- (stop);

\end{tikzpicture}
\caption{Flowchart การสร้างและเรียกใช้ฟังก์ชัน}
\end{figure}

\vspace{7cm}

\subsection{พารามิเตอร์และอาร์กิวเมนต์ (Parameters and Arguments)}

\hspace{1cm}\textlight{พารามิเตอร์ (Parameters) คือตัวแปรที่กำหนดในฟังก์ชัน ส่วนอาร์กิวเมนต์ (Arguments) คือค่าที่ส่งให้ฟังก์ชันเมื่อเรียกใช้ Python มีวิธีการส่งอาร์กิวเมนต์หลายแบบ}

\begin{noticebox}{ประเภทของพารามิเตอร์}

\textlight{ประเภทของพารามิเตอร์ใน Python:}

\begin{itemize}
    \item \textbf{Positional Arguments}: \textcolor{darkgray}{อาร์กิวเมนต์ตามตำแหน่ง}
    \item \textbf{Keyword Arguments}: \textcolor{darkgray}{อาร์กิวเมนต์ตามชื่อ}
    \item \textbf{Default Parameters}: \textcolor{darkgray}{พารามิเตอร์ที่มีค่าเริ่มต้น}
    \item \textbf{*args}: \textcolor{darkgray}{อาร์กิวเมนต์จำนวนไม่จำกัด}
    \item \textbf{**kwargs}: \textcolor{darkgray}{อาร์กิวเมนต์แบบ keyword จำนวนไม่จำกัด}
\end{itemize}

\end{noticebox}

\begin{codebox}{(ตัวอย่าง) ประเภทของพารามิเตอร์ใน Python}
\begin{lstlisting}[style=python]
# Function with default parameters
def introduce(name, age=25, city="Bangkok"):
    print(f"Name: {name}, Age: {age}, City: {city}")

# Different ways to call the function
introduce("Alice")  # Use default values
introduce("Bob", 30)  # Specify age
introduce("Charlie", city="Chiang Mai")  # Use keyword argument
introduce("Diana", 28, "Phuket")  # Specify all values

# Function that accepts unlimited arguments
def sum_all(*numbers):
    total = 0
    for num in numbers:
        total += num
    return total

result1 = sum_all(1, 2, 3)
result2 = sum_all(1, 2, 3, 4, 5)
print("Sum 1:", result1)
print("Sum 2:", result2)
\end{lstlisting}
\end{codebox}

\begin{resultbox}{ผลลัพธ์}
\begin{verbatim}
Name: Alice, Age: 25, City: Bangkok
Name: Bob, Age: 30, City: Bangkok
Name: Charlie, Age: 25, City: Chiang Mai
Name: Diana, Age: 28, City: Phuket
Sum 1: 6
Sum 2: 15
\end{verbatim}
\end{resultbox}

\subsection{ขอบเขตของตัวแปร (Variable Scope)}

\hspace{1cm}\textlight{ขอบเขตของตัวแปร (Variable Scope) กำหนดว่าตัวแปรสามารถเข้าถึงได้จากส่วนไหนของโปรแกรม ใน Python มีขอบเขตหลัก 2 ประเภท คือ ขอบเขตทั่วไป (Global Scope) และขอบเขตท้องถิ่น (Local Scope)}

\begin{noticebox}{ขอบเขตของตัวแปร}

\textlight{ประเภทของขอบเขตตัวแปรใน Python:}

\begin{itemize}
    \item \textbf{Global Scope}: \textcolor{darkgray}{ตัวแปรที่สามารถเข้าถึงได้จากทุกส่วนของโปรแกรม}
    \item \textbf{Local Scope}: \textcolor{darkgray}{ตัวแปรที่สามารถเข้าถึงได้เฉพาะภายในฟังก์ชัน}
    \item \textbf{global keyword}: \textcolor{darkgray}{ใช้เพื่อแก้ไขตัวแปร global ภายในฟังก์ชัน}
    \item \textbf{nonlocal keyword}: \textcolor{darkgray}{ใช้เพื่อแก้ไขตัวแปรในขอบเขตนอกสุด}
\end{itemize}

\end{noticebox}

\begin{codebox}{(ตัวอย่าง) ขอบเขตของตัวแปรใน Python}
\begin{lstlisting}[style=python]
# Global variable
global_var = "I am global"

def test_scope():
    # Local variable
    local_var = "I am local"
    print(f"Inside function - Global: {global_var}")
    print(f"Inside function - Local: {local_var}")

def modify_global():
    global global_var
    global_var = "I am modified global"
    print(f"Modified global: {global_var}")

# Call functions
test_scope()
print(f"Outside function - Global: {global_var}")

modify_global()
print(f"After modification - Global: {global_var}")

# Example of using nonlocal
def outer_function():
    outer_var = "I am outer"
    
    def inner_function():
        nonlocal outer_var
        outer_var = "I am modified outer"
        print(f"Inner function: {outer_var}")
    
    inner_function()
    print(f"Outer function: {outer_var}")

outer_function()
\end{lstlisting}
\end{codebox}

\begin{resultbox}{ผลลัพธ์}
\begin{verbatim}
Inside function - Global: I am global
Inside function - Local: I am local
Outside function - Global: I am global
Modified global: I am modified global
After modification - Global: I am modified global
Inner function: I am modified outer
Outer function: I am modified outer
\end{verbatim}
\end{resultbox}

\subsection{ฟังก์ชัน Built-in ที่สำคัญ (Important Built-in Functions)}

\hspace{1cm}\textlight{Python มีฟังก์ชัน Built-in จำนวนมากที่พร้อมใช้งานทันทีโดยไม่ต้อง import เพิ่มเติม ฟังก์ชันเหล่านี้ช่วยให้นักศึกษาสามารถเขียนโปรแกรมได้อย่างมีประสิทธิภาพมากขึ้น เช่น การแปลงข้อมูล การกรองข้อมูล หรือการคำนวณค่าสถิติพื้นฐาน ซึ่งเป็นทักษะที่จำเป็นในการพัฒนาโปรแกรม}

\begin{codebox}{(ตัวอย่าง) ฟังก์ชัน Built-in ที่สำคัญใน Python}
\begin{lstlisting}[style=python]
# Functions for data type conversion
numbers_str = ["1", "2", "3", "4", "5"]
numbers_int = list(map(int, numbers_str))
print("Converted to int:", numbers_int)

# filter() function - filtering data
def is_even(n):
    return n % 2 == 0

numbers = [1, 2, 3, 4, 5, 6, 7, 8, 9, 10]
even_numbers = list(filter(is_even, numbers))
print("Even numbers:", even_numbers)

# max(), min(), sum() functions
scores = [85, 92, 78, 96, 87]
print("Max score:", max(scores))
print("Min score:", min(scores))
print("Total score:", sum(scores))
print("Average score:", sum(scores) / len(scores))

# sorted() function
names = ["Alice", "Charlie", "Bob", "Diana"]
sorted_names = sorted(names)
print("Sorted names:", sorted_names)

# zip() function
students = ["Alice", "Bob", "Charlie"]
scores = [85, 92, 78]
student_scores = list(zip(students, scores))
print("Student scores:", student_scores)
\end{lstlisting}
\end{codebox}

\begin{resultbox}{ผลลัพธ์}
\begin{verbatim}
Converted to int: [1, 2, 3, 4, 5]
Even numbers: [2, 4, 6, 8, 10]
Max score: 96
Min score: 78
Total score: 438
Average score: 87.6
Sorted names: ['Alice', 'Bob', 'Charlie', 'Diana']
Student scores: [('Alice', 85), ('Bob', 92), ('Charlie', 78)]
\end{verbatim}
\end{resultbox}

\subsection{บททดสอบ}

\begin{exercisebox}{โจทย์ที่ 1: โปรแกรมคำนวณพื้นฐาน}
\textlight{จงเขียนโปรแกรมคำนวณง่ายๆ โดยใช้ Functions}

\textlight{\textbf{ความต้องการ:}}
\begin{itemize}
    \item \textlight{สร้างฟังก์ชัน add(a, b) สำหรับการบวก}
    \item \textlight{สร้างฟังก์ชัน multiply(a, b) สำหรับการคูณ}
    \item \textlight{สร้างฟังก์ชัน square(x) สำหรับการยกกำลังสอง}
    \item \textlight{สร้างฟังก์ชัน calculate\_area(length, width) สำหรับคำนวณพื้นที่สี่เหลี่ยม}
    \item \textlight{เรียกใช้ฟังก์ชันทั้งหมดและแสดงผล}
\end{itemize}

\textlight{\textbf{ตัวอย่าง Output ที่ต้องการ:}}
\begin{codebox}{ผลลัพธ์ที่คาดหวัง}
\begin{lstlisting}[style=python]
Addition: 5 + 3 = 8
Multiplication: 4 * 6 = 24
Square of 7 = 49
Rectangle area (5 x 3) = 15 square units
\end{lstlisting}
\end{codebox}
\textlight{พร้อมสร้าง Flowchart แสดงการทำงานของโปรแกรม}
\end{exercisebox}

\begin{exercisebox}{โจทย์ที่ 2: โปรแกรมจัดการคะแนนง่าย}
\textlight{จงเขียนโปรแกรมจัดการคะแนนโดยใช้ Functions}

\textlight{\textbf{ความต้องการ:}}
\begin{itemize}
    \item \textlight{สร้างฟังก์ชัน calculate\_average(scores) สำหรับคำนวณค่าเฉลี่ย}
    \item \textlight{สร้างฟังก์ชัน find\_highest(scores) สำหรับหาคะแนนสูงสุด}
    \item \textlight{สร้างฟังก์ชัน count\_passed(scores, passing\_score) สำหรับนับจำนวนที่ผ่าน}
    \item \textlight{สร้างฟังก์ชัน get\_grade(score) สำหรับแปลงคะแนนเป็นเกรด A, B, C, D, F}
    \item \textlight{ใช้คะแนนตัวอย่าง [85, 92, 78, 96, 73] และแสดงผลสถิติ}
\end{itemize}

\textlight{\textbf{ตัวอย่าง Output ที่ต้องการ:}}
\begin{codebox}{ผลลัพธ์ที่คาดหวัง}
\begin{lstlisting}[style=python]
Scores: [85, 92, 78, 96, 73]
Average score: 84.8
Highest score: 96
Students passed (>=60): 5
Grade for score 85: B
Grade for score 92: A
Grade for score 78: C
\end{lstlisting}
\end{codebox}
\textlight{พร้อมสร้าง Flowchart แสดงการทำงานของโปรแกรม}
\end{exercisebox}

\vspace{6cm}

\begin{noticebox}{การใช้ Random Module}

\textlight{สำหรับการสุ่มตัวเลขใน Python:}

\begin{itemize}
    \item \textbf{import random}: \textcolor{darkgray}{นำเข้า module สำหรับการสุ่ม}
    \item \textbf{random.randint(a, b)}: \textcolor{darkgray}{สุ่มเลขจำนวนเต็มระหว่าง a ถึง b}
    \item \textbf{random.choice(list)}: \textcolor{darkgray}{สุ่มเลือกสมาชิกจากรายการ}
    \item \textbf{random.random()}: \textcolor{darkgray}{สุ่มเลขทศนิยมระหว่าง 0.0 ถึง 1.0}
\end{itemize}

\end{noticebox}

\begin{exercisebox}{โจทย์ที่ 3: เกมทายตัวเลข}
\textlight{จงเขียนเกมทายตัวเลขง่ายๆ โดยใช้ Functions}

\textlight{\textbf{ความต้องการ:}}
\begin{itemize}
    \item \textlight{สร้างฟังก์ชัน generate\_number() สำหรับสุ่มตัวเลข 1-50}
    \item \textlight{สร้างฟังก์ชัน check\_guess(secret, guess) สำหรับเปรียบเทียบคำตอบ}
    \item \textlight{สร้างฟังก์ชัน get\_hint(secret, guess) สำหรับให้คำแนะนำ}
    \item \textlight{ให้ผู้เล่นทาย 3 ครั้ง}
    \item \textlight{แสดงผลลัพธ์การเล่น}
\end{itemize}

\textlight{\textbf{ตัวอย่าง Output ที่ต้องการ:}}
\begin{codebox}{ผลลัพธ์ที่คาดหวัง}
\begin{lstlisting}[style=python]
Guess the number (1-50): 25
Too high! Try again.
Guess the number (1-50): 15
Too low! Try again.
Guess the number (1-50): 20
Correct! You win in 3 attempts.
\end{lstlisting}
\end{codebox}

\end{exercisebox}

\vspace{9cm}

\section{การวนซ้ำด้วย While Loop}

\hspace{1cm}\textlight{การวนซ้ำด้วย While Loop เป็นโครงสร้างการควบคุมที่สำคัญในการเขียนโปรแกรม ช่วยให้โปรแกรมสามารถทำงานซ้ำๆ ตราบใดที่เงื่อนไขที่กำหนดยังเป็นจริง (True) While Loop เหมาะสำหรับสถานการณ์ที่เราไม่ทราบจำนวนรอบการวนซ้ำล่วงหน้า เช่น การรับข้อมูลจากผู้ใช้จนกว่าจะถูกต้อง หรือการประมวลผลข้อมูลจนครบทุกรายการ ซึ่งเป็นทักษะที่นักศึกษาจะพบและใช้บ่อยในการแก้ปัญหาจริง}

\subsection{โครงสร้างพื้นฐานของ While Loop}

\hspace{1cm}\textlight{While Loop ประกอบด้วยส่วนสำคัญ 3 ส่วน คือ เงื่อนไขการวนซ้ำ บล็อกคำสั่งที่จะทำงานซ้ำ และการอัพเดตตัวแปร การเข้าใจโครงสร้างนี้อย่างถูกต้องจะช่วยให้นักศึกษาเขียนโปรแกรมได้อย่างปลอดภัยและมีประสิทธิภาพ โดยเฉพาะการป้องกันปัญหา Infinite Loop ที่มักเกิดขึ้นกับผู้เริ่มเรียนโปรแกรมมิ่ง}

\begin{noticebox}{โครงสร้างของ While Loop}

\textlight{โครงสร้างพื้นฐานของ While Loop ใน Python:}

\begin{codebox}{}
\begin{lstlisting}[style=python]
while condition:
    # Statements to be repeated
    # Must change variables related to condition
    pass
\end{lstlisting}
\end{codebox}

\textlight{ส่วนประกอบสำคัญ:}
\begin{itemize}
    \item \textbf{while}: \textcolor{darkgray}{คำสั่งสำหรับเริ่มต้นลูป}
    \item \textbf{condition}: \textcolor{darkgray}{เงื่อนไขที่ต้องเป็นจริงเพื่อให้ลูปทำงานต่อ}
    \item \textbf{loop body}: \textcolor{darkgray}{ชุดคำสั่งที่จะทำงานซ้ำ}
    \item \textbf{update}: \textcolor{darkgray}{การเปลี่ยนแปลงตัวแปรเพื่อควบคุมลูป}
\end{itemize}

\end{noticebox}

\begin{codebox}{(ตัวอย่าง) While Loop พื้นฐานใน Python}
\begin{lstlisting}[style=python]
# Example 1: Count from 1 to 5
count = 1
while count <= 5:
    print("Count:", count)
    count += 1

print("Loop finished!")

# Example 2: Sum numbers from 1 to 5
total = 0
number = 1
while number <= 5:
    total += number
    number += 1

print("Total sum:", total)
\end{lstlisting}
\end{codebox}

\begin{resultbox}{ผลลัพธ์}
\begin{verbatim}
Count: 1
Count: 2
Count: 3
Count: 4
Count: 5
Loop finished!
Total sum: 15
\end{verbatim}
\end{resultbox}

\vspace{2cm}

\begin{figure}[H]
\centering
\begin{tikzpicture}[node distance=2cm]

% While loop flowchart
\node (start) [startstop] {Start};
\node (init) [process, below of=start] {count = 1};
\node (condition) [decision, below of=init, yshift=-1.5cm] {count <= 5};
\node (print) [io, below of=condition, yshift=-2cm] {Print "Count:", count};
\node (increment) [process, below of=print] {count += 1};
\node (stop) [startstop, right of=condition, xshift=4cm] {Stop};

\draw [arrow] (start) -- (init);
\draw [arrow] (init) -- (condition);
\draw [arrow] (condition) -- node[anchor=west, xshift=-1.2cm, yshift=0.4cm] {True} (print);
\draw [arrow] (print) -- (increment);
\draw [arrow] (increment.west) -- ++(-3cm,0) |- (condition.west);
\draw [arrow] (condition) -- node[anchor=south, yshift=0.3cm] {False} (stop);

\end{tikzpicture}
\caption{Flowchart ของ While Loop พื้นฐาน}
\end{figure}

\vspace{4cm}

\subsection{การควบคุม While Loop ด้วย break และ continue}

\hspace{1cm}\textlight{นอกจากเงื่อนไขหลักแล้ว Python ยังมีคำสั่งพิเศษสำหรับควบคุมการทำงานของลูปได้อย่างละเอียด ได้แก่ \textbf{break} และ \textbf{continue} การเข้าใจและใช้คำสั่งเหล่านี้อย่างถูกต้องจะช่วยให้นักศึกษาสามารถเขียนโปรแกรมที่มีความยืดหยุ่นและตอบสนองต่อสถานการณ์ต่างๆ ได้ดีขึ้น}

\begin{noticebox}{คำสั่งควบคุมลูป}

\textlight{คำสั่งสำหรับควบคุมการทำงานของลูป:}

\begin{itemize}
    \item \textbf{break}: \textcolor{darkgray}{ออกจากลูปทันที (ไม่ว่าเงื่อนไขจะเป็นจริงหรือเท็จ)}
    \item \textbf{continue}: \textcolor{darkgray}{ข้ามการทำงานที่เหลือในรอบปัจจุบัน และไปรอบถัดไป}
    \item \textbf{else}: \textcolor{darkgray}{ทำงานเมื่อลูปจบแบบปกติ (ไม่ได้ใช้ break)}
    \item \textbf{pass}: \textcolor{darkgray}{ไม่ทำอะไร (ใช้เป็น placeholder)}
\end{itemize}

\end{noticebox}

\begin{codebox}{(ตัวอย่าง) การใช้ break และ continue ใน While Loop}
\begin{lstlisting}[style=python]
# Example of using break
count = 1
while count <= 10:
    if count == 5:
        break  # Stop loop when count = 5
    print(f"Count: {count}")
    count += 1

# Example of using continue
number = 1
while number <= 5:
    if number == 3:
        number += 1
        continue  # Skip number 3
    print(f"Number: {number}")
    number += 1
\end{lstlisting}
\end{codebox}

\begin{resultbox}{ผลลัพธ์}
\begin{verbatim}
Count: 1
...

Number: 1
...
\end{verbatim}
\end{resultbox}

\subsection{While Loop กับการประมวลผลข้อมูล}

\hspace{1cm}\textlight{While Loop มีบทบาทสำคัญในการประมวลผลข้อมูลเมื่อเราไม่ทราบขนาดหรือจำนวนข้อมูลที่แน่นอนล่วงหน้า เช่น การอ่านไฟล์ข้อมูล การรับ input จากผู้ใช้แบบต่อเนื่อง หรือการค้นหาข้อมูลที่ตรงตามเงื่อนไข ทักษะนี้มีความสำคัญมากในการพัฒนาแอปพลิเคชันที่ทำงานกับข้อมูลจริง}

\begin{codebox}{(ตัวอย่าง) While Loop ในการประมวลผลข้อมูล}
\begin{lstlisting}[style=python]
# Example 1: Find average
numbers = [10, 20, 30, 40, 50]
total = 0
index = 0

while index < len(numbers):
    total += numbers[index]
    index += 1

average = total / len(numbers)
print(f"Average: {average}")

# Example 2: Count even numbers
numbers = [1, 2, 3, 4, 5, 6, 7, 8]
even_count = 0
index = 0

while index < len(numbers):
    if numbers[index] % 2 == 0:
        even_count += 1
    index += 1

print(f"Even numbers count: {even_count}")
\end{lstlisting}
\end{codebox}

\begin{resultbox}{ผลลัพธ์}
\begin{verbatim}
Average: 30.0
Even numbers count: 4
\end{verbatim}
\end{resultbox}

\vspace{3cm}

\subsection{While Loop กับ Nested Loops}

\hspace{1cm}\textlight{การซ้อน While Loop เป็นเทคนิคขั้นสูงที่ใช้สำหรับการประมวลผลข้อมูลที่มีโครงสร้างหลายมิติ เช่น การทำงานกับ matrix การสร้างตารางข้อมูล หรือการจำลองสถานการณ์ที่ซับซ้อน แม้จะเป็นเทคนิคที่ต้องใช้ความระมัดระวังในการเขียนเงื่อนไข แต่เป็นทักษะที่สำคัญสำหรับนักศึกษาระดับปริญญาตรี}

\begin{codebox}{(ตัวอย่าง) Nested While Loops ใน Python}
\begin{lstlisting}[style=python]
# Example 1: Create rectangle
row = 1
while row <= 3:
    col = 1
    while col <= 4:
        print("*", end=" ")
        col += 1
    print()  # New line
    row += 1

# Example 2: 2x2 multiplication table
row = 1
while row <= 2:
    col = 1
    while col <= 2:
        result = row * col
        print(f"{row}x{col}={result}", end="  ")
        col += 1
    print()
    row += 1
\end{lstlisting}
\end{codebox}

\begin{resultbox}{ผลลัพธ์}
\begin{verbatim}
* * * * 
* * * * 
* * * * 

1x1=1  1x2=2  
2x1=2  2x2=4  
\end{verbatim}
\end{resultbox}

\vspace{2.5cm}

\subsection{การป้องกัน Infinite Loop}

\hspace{1cm}\textlight{Infinite Loop หรือการวนซ้ำแบบไม่สิ้นสุดเป็นปัญหาที่พบบ่อยและสำคัญในการเขียนโปรแกรม เกิดจากเงื่อนไขที่ไม่เปลี่ยนแปลงหรือไม่มีทางเป็นเท็จได้ การเข้าใจและป้องกันปัญหานี้เป็นทักษะพื้นฐานที่นักศึกษาต้องเรียนรู้ เพื่อให้สามารถเขียนโปรแกรมที่ทำงานได้อย่างมั่นคงและปลอดภัย}

\begin{noticebox}{เทคนิคป้องกัน Infinite Loop}

\textlight{วิธีการป้องกัน Infinite Loop:}

\begin{itemize}
    \item \textbf{ตรวจสอบเงื่อนไข}: \textcolor{darkgray}{ให้แน่ใจว่าเงื่อนไขสามารถเป็นเท็จได้}
    \item \textbf{อัพเดทตัวแปร}: \textcolor{darkgray}{มีการเปลี่ยนแปลงตัวแปรในเงื่อนไขภายในลูป}
    \item \textbf{ใช้ Counter}: \textcolor{darkgray}{กำหนดจำนวนรอบสูงสุดเป็นเงื่อนไขสำรอง}
    \item \textbf{Debug อย่างระมัดระวัง}: \textcolor{darkgray}{ทดสอบลูปด้วยข้อมูลง่ายๆ ก่อน}
\end{itemize}

\end{noticebox}

\begin{codebox}{(ตัวอย่าง) การป้องกัน Infinite Loop}
\begin{lstlisting}[style=python]
# Wrong example - Infinite Loop
# count = 1
# while count <= 5:
#     print(count)
#     # Forgot to increment count, causing infinite loop

# Correct example - Update variables
count = 1
while count <= 5:
    print(f"Count: {count}")
    count += 1  # Update variable to end the loop

# Example using safety counter
attempts = 0
max_attempts = 3
user_input = ""

while user_input != "exit" and attempts < max_attempts:
    user_input = input(f"Enter 'exit' (attempt {attempts + 1}/{max_attempts}): ")
    print(f"You entered: {user_input}")
    attempts += 1

if attempts >= max_attempts:
    print("Maximum attempts reached.")
\end{lstlisting}
\end{codebox}

\begin{resultbox}{ผลลัพธ์}
\begin{verbatim}
Count: 1
Count: 2
Count: 3
Count: 4
Count: 5

Enter 'exit' (attempt 1/3): hello
You entered: hello
Enter 'exit' (attempt 2/3): world
You entered: world
Enter 'exit' (attempt 3/3): exit
You entered: exit
\end{verbatim}
\end{resultbox}

\vspace{6cm}

\subsection{บททดสอบ}

\begin{exercisebox}{โจทย์ที่ 1: โปรแกรมจัดการรายการสินค้าง่าย}
\textlight{จงเขียนโปรแกรมจัดการรายการสินค้าโดยใช้ While Loop}

\textlight{\textbf{ความต้องการ:}}
\begin{itemize}
    \item \textlight{สร้างเมนูแสดงตัวเลือก: 1-Add, 2-Remove, 3-Show, 4-Exit}
    \item \textlight{ใช้ While Loop เพื่อให้โปรแกรมทำงานต่อเนื่องจนกว่าผู้ใช้เลือก Exit}
    \item \textlight{เริ่มต้นด้วยรายการสินค้า: ["apple", "banana"]}
    \item \textlight{เพิ่มสินค้าใหม่เมื่อเลือก Add}
    \item \textlight{ลบสินค้าเมื่อเลือก Remove}
    \item \textlight{แสดงรายการสินค้าทั้งหมดเมื่อเลือก Show}
\end{itemize}

\textlight{\textbf{ตัวอย่าง Output ที่ต้องการ:}}
\begin{codebox}{ผลลัพธ์ที่คาดหวัง}
\begin{lstlisting}[style=python]
=== Shopping List Manager ===
1. Add item
2. Remove item  
3. Show all items
4. Exit
Enter choice (1-4): 3
Current items: ['apple', 'banana']

Enter choice (1-4): 1
Enter item to add: orange
Added: orange

Enter choice (1-4): 3
Current items: ['apple', 'banana', 'orange']

Enter choice (1-4): 4
Goodbye!
\end{lstlisting}
\end{codebox}

\end{exercisebox}

\begin{exercisebox}{โจทย์ที่ 2: เกมทายตัวเลขง่าย}
\textlight{จงเขียนเกมทายตัวเลขโดยใช้ While Loop}

\textlight{\textbf{ความต้องการ:}}
\begin{itemize}
    \item \textlight{สุ่มตัวเลข 1-20 สำหรับให้ผู้เล่นทาย}
    \item \textlight{ใช้ While Loop ให้ผู้เล่นทายได้สูงสุด 5 ครั้ง}
    \item \textlight{ให้คำแนะนำ "Too high!" หรือ "Too low!" หลังแต่ละครั้ง}
    \item \textlight{แสดงจำนวนครั้งที่เหลือหลังแต่ละครั้ง}
    \item \textlight{ถ้าทายถูกให้แสดงผลชนะและจบเกม}
    \item \textlight{ถ้าทายครบ 5 ครั้งแล้วยังไม่ถูกให้แสดงคำตอบ}
\end{itemize}

\textlight{\textbf{ตัวอย่าง Output ที่ต้องการ:}}
\begin{codebox}{ผลลัพธ์ที่คาดหวัง}
\begin{lstlisting}[style=python]
=== Number Guessing Game ===
I'm thinking of a number between 1-20
You have 5 attempts to guess it!

Attempt 1/5 - Enter your guess: 15
Too high! Try again.

Attempt 2/5 - Enter your guess: 8
Too low! Try again.

Attempt 3/5 - Enter your guess: 12
Congratulations! You guessed it in 3 attempts!
The number was 12
\end{lstlisting}
\end{codebox}

\end{exercisebox}

\begin{exercisebox}{โจทย์ที่ 3: โปรแกรมคำนวณคะแนนสอบ}
\textlight{จงเขียนโปรแกรมคำนวณคะแนนสอบโดยใช้ While Loop}

\textlight{\textbf{ความต้องการ:}}
\begin{itemize}
    \item \textlight{ใช้ While Loop รับคะแนนสอบจากผู้ใช้ (ป้อน -1 เพื่อจบ)}
    \item \textlight{ตรวจสอบคะแนนต้องอยู่ระหว่าง 0-100 เท่านั้น}
    \item \textlight{คำนวณค่าเฉลี่ย คะแนนสูงสุด และต่ำสุด}
    \item \textlight{นับจำนวนคะแนนที่ผ่าน (≥60) และไม่ผ่าน (<60)}
    \item \textlight{แสดงผลสรุปเมื่อผู้ใช้ป้อน -1}
    \item \textlight{ถ้าไม่มีคะแนนให้แสดงข้อความเตือน}
\end{itemize}

\textlight{\textbf{ตัวอย่าง Output ที่ต้องการ:}}
\begin{codebox}{ผลลัพธ์ที่คาดหวัง}
\begin{lstlisting}[style=python]
=== Score Calculator ===
Enter scores (0-100), enter -1 to finish:

Enter score: 85
Valid score added: 85

Enter score: 92
Valid score added: 92

Enter score: 150
Invalid! Score must be between 0-100

Enter score: 78
Valid score added: 78

Enter score: -1

=== Summary ===
Total scores: 3
Scores: [85, 92, 78]
Average: 85.0
Highest: 92
Lowest: 78
Passed (>=60): 3
Failed (<60): 0
\end{lstlisting}
\end{codebox}

\end{exercisebox}

\end{document}